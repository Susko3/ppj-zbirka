%% UPUTE ZA AUTORE %%
% - Svaka recenica zadatka u svoj redak (lakse editiranje/diffanje).
% - stil/gramatika/pravopis
%   - izbjegavati "dakanje" -> "je li" / "li"
%   - "nabrojati" -> "nabrojiti"
%   - Njd prevodilac, Gjd prevodioca; Nmn prevodioci, Gmn prevodilaca. Isto vrijedi za sve imenice na -lac.
%   - spojnice npr. LL(1)-gramatika, LR-parsiranje i dr.
%   - koristiti drugo lice množine, a ne u infinitiv ("objasnite algoritam X", a ne "objasniti algoritam X")
% - simboli
%   - nakon tocke koja nije kraj recenice umjesto razmaka staviti ~ (npr.~123)
%   - za epsilon koristiti \varepsilon umjesto \epsilon
%   - za minus uvijek koristiti \(-\), nikako samo -
%   - za tildu koristiti \sim umjesto \tilde{}
% - u alltt okruzenju
%   - za subskript i superskript koristiti \sb{} i \sp{} jer su _ i ^ obicni znakovi
%   - znakove _ i & ne treba prefiksirati s \
% - ostalo
%   - za inline math koristiti \(...\), a ne $...$ (ovo drugo je deprecated TeX nacin)
%   - sve važne nizove u zadatku (npr. niz koji treba leksirati/parsirati) staviti u \texttt{} (text tele-type)
%       - na taj nacin ce biti istaknuti i nece biti prelomljeni
%   - preferirati \texttt{} ispred \verb
%       - If \texttt{your text} produces the same result as \verb+your text+, then there’s no need of \verb in the first place.

%% komentari vezani uz reference na udžbenik/auditorne  %%
% %udzb/str x	gradivo vezano uz zadatak nalazi se na stranici x
% %udzb/str x-y	gradivo vezano uz zadatak nalazi se na stranicama od x do y
% %udzb/str x,y gradivo vezano uz zadatak nalazi se na stranicama x i y
% %aud/prim x	sličan zadatak nalazi se u auditornim vježbama pod brojem x

\documentclass[times, 12pt, utf8]{book}
\RequirePackage[utf8]{inputenc}
\RequirePackage[croatian]{babel}
\usepackage[utf8]{inputenc}
\usepackage{amsmath}
\usepackage{multirow}

\usepackage{alltt}
\renewcommand{\ttdefault}{txtt}

\usepackage{listings}
\usepackage{array}

% ubacuje hiperlinkove u TOC
\usepackage{hyperref}

% omogucuje nastavljanje enumeratea nakon sto se zavrsi
\usepackage{enumitem}

\usepackage[a4paper, top=1.5cm, bottom=1.5cm, left=1.5cm, right=1.5cm]{geometry}
\pagestyle{plain}

% paket za povecanje proreda linija
\usepackage{setspace}

% kod koji brise "Poglavlje" ispred naslova svakog poglavlja
\makeatletter
\renewcommand{\@makechapterhead}[1]{%
\vspace*{50 pt}%
{\setlength{\parindent}{0pt} \raggedright \normalfont
\bfseries\Huge
\ifnum \value{secnumdepth}>1 
   \if@mainmatter\thechapter.\ \fi%
\fi
#1\par\nobreak\vspace{40 pt}}}
\makeatother


% naredba za generiranje opisnika jezicnog procesora
% koristi se sa \JP{Z}{I}{C} za JP pisan jezikom iZgradnje koji prevodi Izvorni jezik u Ciljni jezik
% vim makro za zamijeniti eksplicitni zapis s ovim je (ako sam neki zaboravio)
% f{l"zxf{l"ixf}h"cxF\F\ldf)iJP{"zpa}{"ipa}{"cpa}
% treba pozicionirati kursor u normalnom modu na prvi \ (onaj koji otvara math mode u trenutnom zapisu, ispred JP)
% ili bilo gdje ispred ako izmedju nema znaka {
% makro unistava registre z,i,c i pretpostavlja da su svi jezici oznaceni jednim slovom
\newcommand{\JP}[3]{\(JP_{#1}^{#2 \to #3}\)}

% vim makro za konverziju $...$ u \(...\) sto je preporuceni format za latex math env
% kursor treba postaviti na lijevi $
% xi\(f$xi\)

\title{
\vfill
Skripta sa zadacima za pripremu iz predmeta \\
Prevođenje programskih jezika \\
\vspace{70pt}
}
\author{
\emph{Pripremili:} \\
Ivan Žužak, Ivan Budiselić, Zvonimir Pavlić, \\
Dejan Škvorc, Miroslav Popović, Goran Delač \\}
\date{
\vspace{30pt} 
\emph{Datum posljednje izmjene:} \\
\today \\
\vfill
Fakultet elektrotehnike i računarstva \\
Zavod za elektroniku, mikroelektroniku, inteligentne i računalne sustave
}
\begin{document}
\maketitle

% kod koji mice prazne stranice prije pocetka chaptera ako chapter ne pocinje na parnoj stranici
\let\cleardoublepage\clearpage

\tableofcontents


\chapter*{Predgovor}
\addcontentsline{toc}{chapter}{Predgovor}

\begin{spacing}{1.5}
Ova skripta zadataka namijenjena je kao dopunski nastavni materijal za predmet \href{http://www.fer.unizg.hr/predmet/ppj_a}{Prevođenje programskih jezika} na Fakultetu elektrotehnike i računarstva Sveučilišta u Zagrebu. 
U svojoj PDF inačici, skripta je besplatno dostupna na Web-stranici predmeta.
Skripta je nastala 2011.~godine sakupljanjem i uređivanjem zadataka i teorijskih pitanja iz ispita, međuispita i auditornih vježbi predmeta \emph{Prevođenje programskih jezika} od 2008.~godine nadalje i predmeta \emph{Automati, formalni jezici i jezični procesori 2} od 1999.~do 2008.~godine.
Posebno zahvaljujemo autorima zadataka bez kojih izdavanje ovakve skripte ne bi bilo moguće:
Ivan Budiselić,
Goran Delač,
Ivan Gavran,
Andro Milanović,
Miroslav Popović,
Daniel Skrobo,
Dejan Škvorc,
Ivan Žužak.

Zadaci su u skripti organizirani u poglavlja, pri čemu naslovi poglavlja odgovaraju naslovima poglavlja u udžbeniku \emph{Prevođenje programskih jezika} \cite{udzbenik}.
Uz zadatake je naznačen i dio udžbenika u kojem se obrađuje dio gradiva nužan za rješavanje zadatka i je li sličan zadatak riješen u sklopu auditornih vježbi.
Rad na skripti nastavlja se i dalje, ali i u svom trenutnom obliku ona može biti korisna za savladavanje gradiva predmeta.

Skripta je oblikovana pomoću sustava \LaTeX{}.
S ciljem unaprjeđenja kvalitete skripte, izvorni \LaTeX{} kod skripte javno je dostupan na adresi \url{http://github.com/fer-ppj/ppj-zbirka}.
Molimo korisnike skripte da uočene pogreške ili primjedbe dojave na službenu e-mail adresu predmeta \href{mailto:ppj@zemris.fer.hr}{ppj@zemris.fer.hr} ili na stranicu za primjedbe: \url{http://github.com/fer-ppj/ppj-zbirka/issues}. 

Zahvaljujemo Vedrani Janković na komentarima i prijavama pogrešaka pri pripremi ove inačice skripte.
Na prijavama pogrešaka zahvaljujemo i Dini Šantlu, Marku Đuraseviću te Nini Uzelcu.
\end{spacing}


%%%%%%%%%%%%%%%%%%%%%%%%%%%%%%%%%%%%%%%%%%%%%%%%%%%%%%%%%%%%%%%%%%%%%%%%%%%%%%%

\chapter{Uvod}

%%%%%%%%%%%%%%%%%%%%%%%%%%%%%%%%%%%%%%%%%%%%%%%%%%%%%%%%%%%%%%%%%%%%%%%%%%%%%%%

\begin{enumerate}

% 1999 - 1MI - 2 - (1)
\item
Navedite prednosti i nedostatke uporabe jezičnih procesora. \cite[str.~1]{udzbenik}

% 2000 - 1MI - 10 - (1)
\item
Navedite tri jezika koji su vezani uz definiciju jezičnog procesora. \cite[str.~3]{udzbenik}

% 2001 - 1MI - 2 - (1)
\item
Navedite koja pravila se susreću kod programskih jezika te ih objasnite. \cite[str.~4-11]{udzbenik}

% 2001 - 1MI - 4 - (1)
\item
Objasnite samoprevodioca. \cite[str.~27]{udzbenik}

% 2010 - 1MI - 3 - (1)
\item
Objasnite cilj i osnovne korake postupka samopodizanja. \cite[str.~27]{udzbenik}

% 2002 - 1MI - 2 - (1)
\item
Objasnite što je funkcija preslikavanja jezičnog procesora i navedite njezine vrste. \cite[str.~26]{udzbenik}

% 2003 - 1MI - 4 - (1)
\item
Objasnite podjelu jezičnih procesora s obzirom na broj prolazaka kroz izvorni program. \cite[str.~24-26]{udzbenik}

% 2004 - 1MI - 2 - (1)
\item
Navedite i objasnite korake analize izvornog programa. \cite[str.~3-13]{udzbenik}

% 2007 - 1MI - 2 - (1)
\item
Objasnite razradbu jezičnih procesora s obzirom na dinamiku izvođenja procesa prevođenja. \cite[str.~24-26]{udzbenik}

% 2009 - 1MI - 1 - (1)
% 2003 - 1MI - 2 - (1)
\item
Nabrojite i ukratko objasnite svaki od osnovnih koraka u izgradnji jezičnog procesora. \cite[str.~3]{udzbenik}

% 2009 - P1MI - 2 - (1)
% 2009 - P2MI - 1 - (1)
% 2009 - P3MI - 1 - (1)
% 2009 - P2MI-L - 1 - (1)
\item
Koristeći Hoareov sustav oznaka CSP, opišite vrste jezičnih procesora s obzirom na dinamiku izvođenja. \cite[str.~24]{udzbenik}

% 2010 - 1MI - 4 - (1)
\item
Objasnite kompilatore i interpretatore, odnosno njihovu razliku, ne koristeći Hoareov sustav oznaka CSP. \cite[str.~24-26]{udzbenik}

% 2009 - 3MI - 1 - (1)
% 2000 - 1MI - 2 - (1)
% 2009 - P1MI - 1 - (1)
% 2010 - 1MI - 1 - (1)
% 2010 - P1MI - 1 - (1)
% 2010 - P2MI - 1 - (1)
% 2010 - P1MI-L - 1 - (1)
% 2010 - P2MI-L - 1 - (1)
% 2010 - P3MI - 1 - (1)
% 2010 - P3MI-L - 1 - (1)
\item
Navedite faze rada jezičnog procesora od kojih se sastoje faza analize izvornog programa i faza sinteze ciljnog programa. \cite[str.~3-4]{udzbenik} \cite{auditorne}

% 2000 - 1MI - 1 - (1)
\item
Za svako od tri računala \textit{A}, \textit{B} i \textit{C} s odgovarajućim strojnim jezicima \textit{a}, \textit{b} i \textit{c} izgradite jezične procesore koji prevode jezik \textit{l} u jezik \textit{n} i koji su izvedivi na pojedinim računalima. 
Jezične procesore treba izgraditi pomoću raspoloživih jezičnih procesora \JP{l}{l}{n}, \JP{a}{l}{a}, \JP{b}{n}{o}, \JP{o}{o}{b} i \JP{c}{o}{c}. \cite[str.~27]{udzbenik}
%aud/prim 8

% 2001 - 1MI - 3 - (1)
\item
Za računalo \textit{A} postoji gotov jezični procesor \JP{a}{l}{a}, dok je na računalu \textit{B} na raspolaganju jezični procesor \JP{b}{n}{c}.
Računalo \textit{C} novo je računalo i za njega ne postoji nikakav jezični procesor, a potrebno je program napisan u jeziku \textit{m} prevesti u ciljni program za to računalo.
Raspoloživ je i jezični procesor \JP{m}{m}{n}, a zbog modularnosti je potrebno napisati jezični procesor \JP{?}{m}{l}.
Odredite u kojem je višem jeziku (\textit{l}, \textit{m} ili \textit{n}) potrebno napisati taj jezični procesor da bi se na najkraći mogući način preveo program te navedite najkraći postupak prevođenja programa iz jezika \textit{m} u ciljni jezik \textit{c}. \cite[str.~27]{udzbenik}  \cite{auditorne}
%aud/prim 8

% 2003 - 1MI - 1 - (1)
\item
Tijekom razvoja složenog sustava koji se sastoji od tri računalne arhitekture \textit{A}, \textit{B} i \textit{C} pojavila se potreba za razvojem jezičnog procesora \JP{b}{o}{b}.
Za potrebe sustava već su napisani jezični procesori \JP{a}{l}{a}, \JP{c}{l}{n} i \JP{c}{o}{b}.
Osim toga, preko Interneta su javno dostupna još tri jezična procesora: \JP{l}{m}{b}, \JP{m}{n}{o} i \JP{n}{o}{b}.
Budući da se projekt približava krajnjem roku, nema dovoljno vremena za pisanje novog jezičnog procesora pa je potrebno jezični procesor \JP{b}{o}{b} konstruirati pomoću raspoloživih jezičnih procesora. 
Napišite postupak konstrukcije. \cite[str.~27]{udzbenik} \cite{auditorne}
%aud/prim 8

% 2004 - 3MI - 1 - (1)
% AV - 6 - (1)
\item
Za računalo A postoji jezični procesor \JP{a}{z}{x}, dok je na računalu B dostupan jezični procesor \JP{b}{x}{a}. 
Raspoloživ je i jezični procesor \JP{z}{x}{y}. 
Odredite u kojem višem programskom jeziku (\textit{x}, \textit{y} ili \textit{z}) treba izgraditi jezični procesor \JP{?}{y}{b}, tako da se može ostvariti prevođenje programa napisanog u jeziku x u ciljni jezik b. 
Navedite sve korake u postupku prevođenja programa. \cite[str.~27]{udzbenik} \cite{auditorne}
%aud/prim 8

% 2007 - 1MI - 6 - (1)
\item
Računalni sustav sastoji se od jednog računala arhitekture \textit{A} i jednog računala arhitekture \textit{B}.
Za računalnu arhitekturu \textit{B} razvijen je jezični procesor \JP{b}{c}{b} kojim je omogućeno izvođenje programa napisanih višim programskim jezikom \textit{c} na računalu arhitekture \textit{B}.
Osim toga, na raspolaganju su jezični procesori \JP{p}{c}{b}, \JP{p}{p}{b}, \JP{c}{p}{a} i \JP{c}{c}{b}.
Uporabom navedenih jezičnih procesora izgradite jezični procesor koji će na računalu arhitekture \textit{A} omogućiti prevođenje programa napisanih višim programskim jezikom \textit{p} u programe izvodive na računalu arhitekture \textit{B}. \cite[str.~27]{udzbenik} \cite{auditorne}
%aud/prim 8

% 2009 - 1MI - 8 - (1)
% 2010 - 1MI - 9 - (1)
\item
Prikažite postupak izgradnje izvodivog jezičnog procesora koji prevodi jezik \textit{l} u strojni jezik \textit{a}.
Na raspolaganju su samoprevodilac koji prevodi jezik \textit{l} u strojni jezik \textit{a}, \JP{a}{p}{s}, \JP{a}{s}{a}, \JP{b}{l}{q}, \JP{b}{r}{a} i \JP{r}{q}{p} te računala \textit{A} i \textit{B} na kojima se mogu izvoditi programi pisani strojnim jezicima \textit{a} i \textit{b}. \cite[str.~27]{udzbenik} \cite{auditorne}
%aud/prim 8

% 2009 - P1MI - 8 - (1)
% 2009 - P2MI - 6 - (1)
% 2009 - P2MI-L - 4 - (1)
% 2010 - P1MI - 7 - (1)
% 2002 - 1MI - 1 - (1)
\item
Tijekom razvoja složenog sustava koji se sastoji od dvije računalne arhitekture \textit{A} i \textit{B} pojavila se potreba za razvojem jezičnog procesora \JP{b}{l}{a}.
Za potrebe sustava već je napisan jezični procesor \JP{a}{m}{a}.
Osim toga, preko Interneta su javno dostupna još četiri jezična procesora: \JP{m}{l}{b}, \JP{l}{m}{a}, \JP{l}{l}{a} i \JP{m}{m}{b}.
Budući da se projekt približava krajnjem roku, nema dovoljno vremena za pisanje novog jezičnog procesora pa je potrebno jezični procesor \JP{b}{l}{a} konstruirati pomoću raspoloživih jezičnih procesora.
Napišite postupak konstrukcije \JP{b}{l}{a}. \cite[str.~27]{udzbenik} \cite{auditorne}
%aud/prim 8

% 1999 - 1MI - 3 - (1)
\item
Za novo računalo “Amiga 2001” (računalo \textit{C}) kao osnovni jezik treba upotrijebiti “C++” (jezik \textit{m}) pa je potrebno izgraditi jezični procesor \JP{c}{m}{c}.
Na “PC” računalu (računalo \textit{B}) postoji jezični procesor \JP{b}{l}{b}, koji prevodi jezik “C” (jezik \textit{l}) u Intelov strojni jezik te jezični procesor \JP{b}{k}{c}, koji prevodi jezik “BCPL” (jezik \textit{k}) u strojni jezik računala “Amiga 2001”.
Na raspolaganju je i računalo “Amiga 4000” (računalo \textit{A}) sa pripadnim jezičnim procesorom \JP{a}{k}{a}, koji prevodi jezik “BCPL” u Motorolin strojni jezik.
Pored ovoga postoje i jezični procesori \JP{k}{l}{k} (kros-kompilator) koji prevodi jezik “C” u jezik “BCPL” i koji je napisan u “BCPL”-u te \JP{l}{m}{c} koji prevodi “C++” u strojni jezik računala “Amiga 2001”, a napisan je u “C”-u.
Napišite postupak samopodizanja kojim se može dobiti traženi jezični procesor uz uporabu postojećih jezičnih procesora. \cite[str.~27]{udzbenik} \cite{auditorne}
%aud/prim 8

\end{enumerate}

%%%%%%%%%%%%%%%%%%%%%%%%%%%%%%%%%%%%%%%%%%%%%%%%%%%%%%%%%%%%%%%%%%%%%%%%%%%%%%%

\chapter{Leksička analiza}

%%%%%%%%%%%%%%%%%%%%%%%%%%%%%%%%%%%%%%%%%%%%%%%%%%%%%%%%%%%%%%%%%%%%%%%%%%%%%%%

\begin{enumerate}[resume]

% 1999 - 1MI - 6 - (2)
\item
Opišite dinamiku izvođenja leksičke analize. \cite[str.~52]{udzbenik}

% 1999 - 1MI - 8 - (2)
\item 
Opišite program simulator leksičkog analizatora zasnovan na tablici prijelaza \(\varepsilon\)-NKA. \cite[str.~60-62]{udzbenik}

% 1999 - 3MI - 1 - (2)
\item 
Opišite način izrade globalne tablice znakova. \cite[str.~50-52]{udzbenik}

% 2001 - 1MI - 7 - (2)
\item
Nabrojite i objasnite osnovne klase leksičkih jedinki. \cite[str.~46-49]{udzbenik}

% 2002 - 1MI - 4 - (2)
\item
Opišite program simulator leksičkog analizatora zasnovan na tablici prijelaza DKA. \cite[str.~58-60]{udzbenik}

% 2004 - 1MI - 4 - (2)
\item
Objasnite kako leksički analizator utvrđuje leksičku pogrešku i navedite dva postupka oporavka od pogreške. \cite[str.~58]{udzbenik}

% 2004 - 1MI - 7 - (2)
\item
Navedite namjenu programa Lex i opišite strukturu ulazne datoteke Lexa. \cite[str.~64-70]{udzbenik}

% 2007 - 1MI - 4 - (2)
\item
Opišite podatkovne strukture leksičkog analizatora. \cite[str.~49-52]{udzbenik}

% 2007 - 1MI - 7 - (2)
\item
Objasnite metode opisa leksičkih jedinki, pravila određivanja klasa leksičkih jedinki i postupak grupiranja leksičkih jedinki koje se koriste za izgradnju generatora leksičkog analizatora. \cite[str.~56-58]{udzbenik}

% 2009 - 1MI - 2 - (2)
% 2010 - P1MI-L - 2 - (2)
\item
Navedite i ukratko objasnite pet zadataka leksičkog analizatora. \cite[str.~44-45]{udzbenik}

% 2009 - 2MI - 1 - (2)
\item
Navedite i na primjeru objasnite dva osnovna načina razrješavanja nejednoznačnosti u leksičkoj analizi. \cite[str.~63-64]{udzbenik}

% 2009 - 1MI - 3 - (2)
% 2009 - P1MI - 4 - (2)
% 2009 - P2MI-E - 1 - (2)
% 2010 - 1MI - 5 - (2)
\item
Navedite i objasnite varijable koje koristi simulator zasnovan na tablici prijelaza DKA.
U ovisnosti o navedenim varijablama, objasnite postupak simulatora za grupiranje i određivanje klase leksičke jedinke. \cite[str.~58-69]{udzbenik}
% 2010 - 2MI - 2 - (2)
% 2010 - P1MI-L - 5 - (2)

\item
Objasnite postupak razrješenja nejednoznačnosti u leksičkoj analizi pretraživanjem lijevog konteksta. \cite[str.~63-64]{udzbenik}

% 2000 - 1MI - 6 - (2)
% 2010 - P1MI-L - 5 - (2)
\item
Objasnite postupak razrješenja nejednoznačnosti pretraživanjem desnog konteksta. \cite[str.~63-64]{udzbenik}

% 2010 - P1MI - 2 - (2)
% 2010 - P1MI-L - 4 - (2)
\item
Navedite pravila za određivanje klase leksičke jedinke i grupiranje znakova u leksičke jedinke. \cite[str.~57-58]{udzbenik}

% 2010 - P1MI - 3 - (2)
% 2010 - P1MI-L - 3 - (2)
\item
Navedite ulaze i izlaze iz generatora leksičkog analizatora i leksičkog analizatora ako je leksički analizator ostvaren kao zasebni prolaz jezičnog procesora. \cite[str.~53]{udzbenik}

% 2010 - P1MI - 4 - (2)
\item
Navedite strukture podataka pogodne za ostvarenje tablice znakova i asimptotsku složenost osnovnih operacija nad tablicom znakova za svaku predloženu strukturu podataka. \cite[str.~50-52]{udzbenik}

% 2010 - P1MI - 5 - (2)
\item
Opišite algoritam leksičkog analizatora zasnovanog na tablici prijelaza DKA. \cite[str.~58-60]{udzbenik}

% 1999 - 1MI - 9 - (2)
\item 
Za zadani program nacrtajte sve tablice koje se stvaraju u leksičkoj analizi.
Ključne riječi označene su masnim slovima. \cite[str.~51]{udzbenik} \cite{auditorne}

\begin{alltt}
\textbf{program} Phoebe;
  j:=1; k:=1;
  ispiši (j,k);
  \textbf{za} i:=3 \textbf{do} 20 \bf{čini}
    k:=j+k;
    j:=k-j;
    ispiši (k);
  \textbf{kraj}
\textbf{kraj}
\end{alltt}
%aud/prim 2

% 2000 - 1MI - 4 - (2)
\item
Zadan je program u jeziku čije su ključne riječi, operatori i specijalni znakovi: \verb|main|, \verb|if|, \verb|for|, \verb|(|, \verb|)|, \verb|{|, \verb|}|, \verb|;|, \verb|>|, \verb|<|, \verb|=|, \verb|+|, \verb|-|, \verb|*|.
Ispišite sve tablice koje su izlaz leksičkog analizatora. \cite[str.~51]{udzbenik} \cite{auditorne}

\noindent
\begin{alltt}
main()
\verb|{|
  a23=c+b57*27-3*a5;
  if(a>77)
  \verb|{|
    a=a23;
    a=a+c;
  \verb|}|
  for(i=0;i<12;i=i+1)
  \verb|{|
    a=a+i;
    b57=a5-3.7*a5;
    i=i+1;
  \verb|}|
\verb|}|
\end{alltt} 
%aud/prim 2

% 2001 - 2MI - 10 - (2)
\item
Konstruirajte sve izlazne tablice leksičkog analizatora za dani izvorni program.
Ključne riječi u programu masno su otisnute, a konstante su u kurzivu. \cite[str.~51]{udzbenik} \cite{auditorne}

\begin{alltt}
\textbf{import} javax.swing.*;
\textbf{public class} FrameDemo \verb|{|
  \textbf{public static void} main(String[] args) \verb|{|
    JFrame jframe=\textbf{new} JFrame(\textit{"Example"});
    jframe.setSize(\textit{400},\textit{100});
    jframe.setVisible(\textit{true});
  \verb|}|
\verb|}|
\end{alltt} 
%aud/prim 2

% 2002 - 1MI - 3 - (2)
\item
Za zadani isječak kôda napišite sadržaje svih tablica leksičkog analizatora.
Ključne su riječi masno otisnute. \cite[str.~51]{udzbenik} \cite{auditorne}

\begin{alltt}
\textbf{if} (!fread(&id32, \textbf{sizeof}(\textbf{struct} ID3v2_Header),1,fp) ||
    strncmp(id32.id,"ID3",3))
  \textbf{return} 0;
\end{alltt} 
%aud/prim 2

% 2004 - 2MI - 3 - (2)
\item
Napišite sve izlazne tablice simulatora leksičkog analizatora programskog jezika C za zadani isječak ulaznog programa.
Pretpostavite da leksički analizator uspješno grupira višeznakovne operatore. \cite[str.~51]{udzbenik} \cite{auditorne}

\begin{tabular}{ | c | c | c | c | c | c | c | c | c | c | c | c | c | c | c | c | } \hline
Pozicija & 1 & 2 & 3 & 4 & 5 & 6 & 7 & 8 & 9 & 10 & 11 & 12 & 13 & 14 & 15 \\ \hline
Znak & i & f & ( & i & f & 3 & = & = & 2 & ) & b & 2 & + & + & ; \\ \hline
\end{tabular}

\begin{tabular}{ | c | c | c | c | c | c | c | c | c | c | c | } \hline
Pozicija & 16 & 17 & 18 & 19 & 20 & 21 & 22 & 23 & 24 & 25 \\ \hline
Znak & e & l & s & e & b & 2 & + & = & 2 & ; \\ \hline
\end{tabular}
%aud/prim 2

% 2007 - 1MI - 1 - (2)
\item
Leksički analizirajte zadani programski odsječak te konstruirati sve izlazne tablice leksičkog analizatora.
Ključne su riječi masno otisnute. \cite[str.~51]{udzbenik} \cite{auditorne}

\begin{alltt}
\textbf{double} factor = 0.356E-6;
\textbf{int} f1(\textbf{const} \textbf{double} &x) \verb|{|
  \textbf{if} (!(x*factor< temp.treshold))
    \textbf{return} temp.high;
  \textbf{return} temp.low;
\verb|}|
\end{alltt} 
%aud/prim 2

% 2009 - P3MI - 2 - (2)
% 2003 - 2MI - 1 - (2)
\item
Konstruirajte sve izlazne tablice leksičkog analizatora za zadani programski odsječak napisan u programskom jeziku C.
Ključne su riječi masno otisnute. \cite[str.~51]{udzbenik} \cite{auditorne}

\begin{alltt}
clock_t clock (\textbf{void})
\verb|{|
  \textbf{struct} timeb now;
  clock_t elapsed;
 
  /* Calculate the difference between the initial time and now. */
 
  ftime(&now);
  elapsed=(now.time-_itimeb.time)*20;
  \textbf{return}(elapsed);
\verb|}|
\end{alltt} 
%aud/prim 2

% 2010 - P1MI-L - 7 - (2)
\item
Leksički analizirajte zadani programski odsječak te konstruirati sve izlazne tablice leksičkog analizatora.
Ključne su riječi podvučene, a sve su linije koje započinju znakom \texttt{\#} komentari. \cite[str.~51]{udzbenik} \cite{auditorne}

\begin{alltt}
\underline{def} format_cond(string_list, c):
  # funkcija formatira ulaznu listu string_list ovisno o vrijednosti
  # parametra c
  format_func = \underline{lambda} s: " ".join(s.split())
  \underline{for} s \underline{in} string_list:
    rez.append(format_func(s))
  \underline{return} rez
\end{alltt} 
%aud/prim 2

% AV - 1 - (2)
\item
Leksički analizirajte zadani programski odsječak te konstruirati sve izlazne tablice leksičkog analizatora.
Ključne su riječi podvučene. \cite[str.~51]{udzbenik} \cite{auditorne}

\begin{alltt}
\underline{const} \underline{string} Instrument = "Gitara";
\underline{enum} Padez \verb|{| Nominativ, Genetiv, Dativ, Akuzativ \verb|}|;
\underline{int} a = (\underline{int}) Padez.Nominativ + (\underline{int}) Padez. Akuzativ;
Instrument += "Klasicna";
\end{alltt} 
%aud/prim 2

% 2000 - 2MI - 1 - (2)
% 2010 - 1MI - 6 - (2)
\item
Za sljedeći niz izgradite tablicu uniformnih znakova, tablicu identifikatora i tablicu konstanti: \texttt{AKOI>31ONDAA++;INACEA+=4;}

Ključne riječi i operatori, identifikatori te konstante definirani su izrazima: 
\begin{alltt}
KROS := < | > | AKO | ONDA | INACE | + | - | = | ;
IDN := slovo ( slovo | brojka )*
KON := (brojka)+
\end{alltt}

Slovo i brojka redom označavaju velika i mala slova abecede te znamenke dekadskog sustava.
Za razrješavanje nejednoznačnosti u grupiranju i određivanju klase leksičke jedinke koristite pravilo grupiranja najduljeg prefiksa koji je definiran barem jednim regularnim izrazom.
Za prefikse jednake duljine, prednost ima izraz koji je zadan prvi. \cite[str.~51]{udzbenik} \cite{auditorne}
%aud/prim 2

% 2001 - 1MI - 10 - (2)
\item
Ostvaren je program simulator leksičkog analizatora zasnovan na tablici prijelaza DKA s jednostavnim postupkom oporavka od pogreške.
Simulator prepoznaje dva niza: \texttt{AUTO} i \texttt{AUTOMOBIL}.
Na ulazu automata pojavljuje se niz \texttt{AUTOMATSKIAUTOMOBIL}.
Odredite koje će nizove simulator leksičkog analizatora prepoznati i hoće li ispisati neke greške.
Potrebno je i ispisati tablicu stanja unutarnjih kazaljki (početak, završetak i posljednji) programa simulatora za svaki učitani znak. \cite[str.~58-60]{udzbenik}

% 2003 - 1MI - 3 - (2)
\item
Generatoru leksičkog analizatora potrebno je zadati pravila za analizu Booleovih izraza.
Booleovi se izrazi sastoje od identifikatora (počinju slovom), Booleovih konstanti (\texttt{TRUE} i \texttt{FALSE}), operatora (\texttt{\&}, \texttt{\(\vert\)}, \texttt{\(\hat{\ }\)}, \texttt{\(\sim\)} i \texttt{=}) te okruglih zagrada.
Napišite sve regularne izraze i akcije simulatora. \cite[str.~47-48]{udzbenik} \cite{auditorne}
%aud/prim 3

% 1999 - 1MI - 1 - (2)
\item
U simulatoru leksičkog analizatora za aritmetičke izraze, potrebno je razriješiti nejednoznačnost uzrokovanu uporabom unarnih operatora.
Aritmetički izrazi sastoje se od identifikatora (počinju slovom), cjelobrojnih konstanti, operatora \{\texttt{+}, \texttt{\(-\)}, \texttt{*}, \texttt{/} i \texttt{=}\} te okruglih zagrada.
Napišite sve regularne izraze i akcije simulatora. \cite[str.~47-48]{udzbenik} \cite{auditorne}
%aud/prim 3

% 2010 - P1MI - 10 - (2)
% 2010 - P2MI - 8 - (2)
\item
Zadan je jezik L koji sadrži cjelobrojne aritmetičke izraze.
Leksičke su jedinke jezika L varijable, konstante, binarni operatori \{\texttt{+}, \texttt{\(-\)}, \texttt{*}, \texttt{/}\}, okrugle zagrade i unarni operatori \{\texttt{\(-\)}, \texttt{++}, \texttt{\(--\)}\}.
Varijable se sastoje od slova i brojki te moraju započinjati slovom, a konstante su cjelobrojne.
Unarni operator \texttt{\(-\)} označava negaciju varijable ili konstante.
Unarni operatori \texttt{++} i \texttt{\(--\)} imaju značenje kao u jeziku C.
Definirajte pravila leksičkog analizatora kojima se ulazni niz rastavlja na leksičke jedinke. \cite[str.~47-48]{udzbenik}

% 2009 - 1MI - 7 - (2)
\item
U postupku sintaksne analize programskog jezika X potrebno je razlikovati dekadske, oktalne i heksadekadske pozitivne cjelobrojne konstante.
Napišite regularne izraze koji će omogućiti ispravno određivanje klase cjelobrojne konstante u leksičkoj analizi.
Oktalne konstante započinju znamenkom \texttt{0} (npr. \texttt{0134}, \texttt{071} i \texttt{00032}).
Dopušteno je da dekadske konstante započinju vodećim nulama ako sadrže barem jednu znamenku \texttt{8} ili \texttt{9} (dekadske su konstante npr. \texttt{00039}, \texttt{488} i \texttt{455}).
Heksadekadske konstante započinju nizom 0x i dalje sadrže dekadske znamenke i mala slova \texttt{a}, \texttt{b}, \texttt{c}, \texttt{d}, \texttt{e}, \texttt{f}  (npr. \texttt{0x13a}, \texttt{0x043f} i \texttt{0x00fed}). \cite[str.~47-48]{udzbenik}

% 2009 - 2MI - 7 - (2)
\item
Prikažite postupak obrade i izlaz leksičkog analizatora zasnovanog na regularnim izrazima iz tablice na sljedećim ulaznim nizovima (obrada svakog niza je nezavisna): \cite[str.~47-48]{udzbenik} \cite{auditorne}\\
i. aabab\\
ii. ababbba\\
iii. abababc

\begin{tabular}{|c|c|c|} \hline
r1 & aab(c)* & ispiši("r1") \\ \hline
r2 & (a)*b & ispiši("r2") \\ \hline
r3 & abab & ispiši("r3") \\ \hline
r4 & ab / c & ispiši("r4") \\ \hline
r5 & ababb & uđi u stanje S; ODBACI; \\ \hline
r6 & bbb & ispiši("r6") \\ \hline
r7 & $<$S$>$ bba & ispiši("r7"); izađi iz stanja S; \\ \hline
r8 & (c)* & ispiši("r8") \\ \hline
\end{tabular}
%aud/prim 3

% 2010 - 1MI - 8 - (2)
\item
Prikažite postupak obrade i izlaz leksičkog analizatora zasnovanog na regularnim izrazima iz tablice na sljedećim ulaznim nizovima (obrada svakog niza je nezavisna):\\
1) aababcaaab\\
2) ccababbba

Prikažite koji su znakovi ulaznih nizova uspješno grupirani pomoću kojih regularnih izraza. Nije potrebno prikazivati rad analizatora u smislu varijabli koje koristi analizator zasnovan na tablici prijelaza DKA ili \(\varepsilon\)-NKA. \cite[str.~47-48]{udzbenik} \cite{auditorne}

\begin{tabular}{|c|c|c|} \hline
r1 & aab(c)* & ispiši("r1") \\ \hline
r2 & (a)*b & ispiši("r2") \\ \hline
r3 & abab & ispiši("r3") \\ \hline
r4 & ab / c & ispiši("r4") \\ \hline
r5 & ababb & uđi u stanje S; ODBACI; \\ \hline
r6 & bbb & ispiši("r6") \\ \hline
r7 & $<$S$>$ bba & ispiši("r7"); izađi iz stanja S; \\ \hline
r8 & (c)* & ispiši("r8") \\ \hline
\end{tabular} 
%aud/prim 3

% 2010 - P1MI - 8 - (2)
% 2010 - P1MI-L - 8 - (2)
% 2010 - P2MI-L - 8 - (2)
% 2010 - P3MI-L - 8 - (2)
\item
Prikažite postupak obrade i izlaz leksičkog analizatora zasnovanog na regularnim izrazima iz tablice na sljedećim ulaznim nizovima (obrada svakog niza je nezavisna):\\
1) aab\\
2) aaa\%\%b\\
3) b\#\%\%

\begin{tabular}{|c|c|c|} \hline
r1 & \%\%b & ispiši("r1") \\ \hline
r2 & \%\%ba* & ispiši("r2") \\ \hline
r3 & a & ispiši("r3") \\ \hline
r4 & aa / b & ispiši("r4") \\ \hline
r5 & aaa & ispiši("r5") \\ \hline
r6 & b & ispiši("r6") \\ \hline
r7 & b(\%$\mid$\#) & uđi u stanje S; ODBACI \\ \hline
r8 & $<$S$>$ \#\%\% & ispiši("r8"); izađi iz stanja S \\ \hline
\end{tabular}

Prikažite koji su znakovi ulaznih nizova uspješno grupirani pomoću kojih regularnih izraza.
Nije potrebno prikazivati rad analizatora u smislu varijabli koje koristi analizator zasnovan na tablici prijelaza DKA ili \(\varepsilon\)-NKA. \cite[str.~47-48]{udzbenik} \cite{auditorne}
%aud/prim 3

% AV - 2 - (2)
\item
Na osnovi navedenih pravila odredite i objasnite izlaz leksičkog analizatora za nizove a), b) i c). \cite[str.~47-48]{udzbenik} \cite{auditorne}\\
a) yyy++x\\
b) yyx\\
c) x!++

\begin{tabular}{|c|c|c|} \hline
r1 & ++x & ispiši("r1") \\ \hline
r2 & ++xy* & ispiši("r2") \\ \hline
r3 & y & ispiši("r3") \\ \hline
r4 & yy / x & ispiši("r4") \\ \hline
r5 & yyy & ispiši("r5") \\ \hline
r6 & x & ispiši("r6") \\ \hline
r7 & x(+ \(\mid\) !) & uđi u stanje S; ODBACI \\ \hline
r8 & \(<\)S\(>\) !++ & ispiši("r8"); izađi iz stanja S \\ \hline
\end{tabular}
%aud/prim 3

% 2010 - 1MI - 7 - (2)
\item
Neka su zadana sljedeća pravila prevođenja u ulaznoj datoteci za program LEX:

\begin{alltt}
\%\%
[a-z]+	;
[0-4]+	;
\end{alltt}

Za sljedeći ulazni niz prikažite izlaz leksičkog analizatora generiranog programom LEX za prethodno navedena leksička pravila: \texttt{Aaa5+Bbb4+ccc+DDD=91} \cite[str.~64-70]{udzbenik}

% 2010 - P1MI - 9 - (2)
% 2010 - P1MI-L - 9 - (2)
% 2010 - P3MI - 7 - (2)
% 2009 - P1MI - 5 - (2)
% 2009 - 1MI - 9 - (2)
\item
Za leksički analizator zasnovan na zadanom \(\varepsilon\)-NKA M = (\{\(q_{0}\), \(q_{1}\), \(q_{2}\), \(q_{3}\), \(q_{4}\)\}, \{\(a, b\)\}, \(\delta\), \(q_{0}\), \{\(q_{3}\), \(q_{4}\)\}) po koracima prikažite postupak računanja skupa \(\varepsilon\)-OKRUŽENJE(\(\delta\)(\{\(q_{0}\), \(q_{1}\)\}, \(a\))) koristeći stog i dva bit-vektora. \cite[str.~60-62]{udzbenik}

\begin{tabular}{|c|c|c|c|} \hline

\(\delta\) & a & b & \(\varepsilon\) \\ \hline
\(q_{0}\) & \(\{q_{0}, q_{2}\}\) & \(\{q_{1}, q_{3}\}\) & \(\{q_{1}, q_{2}\}\) \\ \hline
\(q_{1}\) & \(\{q_{3}\}\) & - & \(\{q_{0}, q_{3}\}\) \\ \hline
\(q_{2}\) & - & - & \(\{q_{3}\}\) \\ \hline
\(q_{3}\) & \(\{q_{1}, q_{4}\}\) & - & - \\ \hline
\(q_{4}\) & - & \(\{q_{3}, q_{4}\}\) & - \\ \hline
\end{tabular} 

\end{enumerate}

%%%%%%%%%%%%%%%%%%%%%%%%%%%%%%%%%%%%%%%%%%%%%%%%%%%%%%%%%%%%%%%%%%%%%%%%%%%%%%%

\chapter{Sintaksna analiza}

%%%%%%%%%%%%%%%%%%%%%%%%%%%%%%%%%%%%%%%%%%%%%%%%%%%%%%%%%%%%%%%%%%%%%%%%%%%%%%%

\begin{enumerate}[resume]

% 2009 - 2MI - 3 - (3)
% 2010 - P2MI - 3 - (3)
\item
Definirajte relacije \emph{IspodZnaka} i \emph{ReduciranZnakom} za parsiranje tehnikom \emph{Pomakni-Pronađi}. \cite[str.~121-123]{udzbenik}

% 2010 - 1MI - 2 - (3)
\item
Navedite uvjete pod kojima je kontekstno neovisna gramatika ujedno i S-gramatika. \cite[str.~87]{udzbenik}

% 2010 - P2MI-L - 4 - (3)
% 2010 - 2MI - 3 - (3)
\item
Navedite uvjete pod kojima je kontekstno neovisna gramatika ujedno i LL(1)-gramatika. \cite[str.~98]{udzbenik}

% 1999 - 2MI - 2 - (3)
\item 
Navedite postupak izgradnje kanonskog LR-parsera na temelju izgrađenog DKA. \cite[str.~150-151]{udzbenik}

% 1999 - 2MI - 4 - (3)
\item 
Objasnite postupak određivanja relacija prednosti na temelju zadane gramatike. \cite[str.~133-135]{udzbenik}

% 2009 - P2MI - 4 - (3)
% 2009 - P2MI-L - 2 - (3)
\item
U pseudokodu sličnom jeziku C napišite algoritam parsiranja tehnikom prednosti operatora. \cite[str.~130-133]{udzbenik}

% 2004 - 1MI - 9 - (3)
\item
Navedite i kratko opišite postupke pretvorbe produkcija u produkcije LL(1)-gramatike. \cite[str.~107-111]{udzbenik}

% 1999 - 2MI - 6 - (3)
% 2010 - 3MI - 4 - (3)
\item
Objasnite postupak lijevog izlučivanja za preuređivanje produkcija gramatike. \cite[str.~108]{udzbenik}

% 2003 - 1MI - 9 - (3)
\item
Objasnite postupak uklanjanja lijeve rekurzije tijekom pretvorbe produkcija u LL(1) oblik. \cite[str.~110-111]{udzbenik}

% 1999 - 2MI - 8 - (3)
\item 
Navedite algoritam za izračunavanje ZAPOČINJE skupova za produkcije. \cite[str.~102-103]{udzbenik}

% 2000 - 2MI - 9 - (3)
\item
Opišite algoritam za izračunavanje relacije Ispred. \cite[str.~105-107]{udzbenik}

% 2009 - 2MI - 2 - (3)
% 2010 - P2MI - 4 - (3)
\item
Navedite korake u računanju PRIMIJENI skupova za produkcije. \cite[str.~107]{udzbenik}

% 2002 - 1MI - 9 - (3)
\item
Opišite postupak računanja skupova SLIJEDI za prazne nezavršne znakove. \cite[str.~107]{udzbenik}

% 2000 - 1MI - 8 - (3)
% 2009 - P1MI - 3 - (3)
% 2009 - P2MI - 2 - (3)
\item
Navedite i kratko opišite podatkovnu strukturu sintaksnog analizatora. \cite[str.~80]{udzbenik}

% 2000 - 2MI - 2 - (3)
% 2004 - 2MI - 4 - (3)
\item
Opišite kako se izvodi nadziranje i oporavak od pogrešaka kod LR-parsiranja. \cite[str.~153-154]{udzbenik}

% 2000 - 2MI - 4 - (3)
\item
Objasnite parsiranje od dna prema vrhu metodom \emph{Pomakni-Reduciraj}. \cite[str.~126-130]{udzbenik}

% 2010 - P2MI - 5 - (3)
% 2007 - 2MI - 9 - (3)
\item
Navedite i definirajte korake algoritma izgradnje kanonskog LR(1)-parsera. \cite[str.~147-151]{udzbenik}

% 2010 - 3MI - 5 - (3)
% 2001 - 1MI - 9 - (3)
\item
Opišite postupak izgradnje potisnog automata za S-gramatiku. \cite[str.~87-88]{udzbenik}

% 2001 - 2MI - 4 - (3)
\item
Definirajte LL(1)-gramatiku i kratko opišite konstrukciju potisnog automata za LL(1)-gramatiku. \cite[str.~95-99]{udzbenik}

% 2010 - P2MI-L - 2 - (3)
\item
Navedite zadatke sintaksnog analizatora. \cite[str.~71]{udzbenik}

% 2010 - P2MI-L - 3 - (3)
\item
Navedite ulaze i izlaze iz sintaksnog analizatora ako je sintaksni analizator ostvaren kao zasebni prolaz jezičnog procesora. \cite[str.~71]{udzbenik}

% 2010 - P2MI-L - 5 - (3)
\item
Objasnite pojmove parsiranje, parsiranje od dna prema vrhu i parsiranje od vrha prema dnu. \cite[str.~84,113]{udzbenik}

% 2010 - 3MI - 1 - (3)
\item
Neovisno poredajte gramatike LL(1), S i Q te gramatike LALR(1), SLR(1), LR(0) i LR(1) uzlazno po općenitosti. \cite[str.~71-154]{udzbenik}

% 2010 - 2MI - 4 - (3)
% 2007 - 2MI - 4 -(3)
\item
Objasnite namjenu programa Yacc te dijelova ulazne datoteke za program Yacc. \cite[str.~154-158]{udzbenik}

% 2010 - 2MI - 1 - (3)
\item
Ukratko objasnite postupak oporavka od pogreške kod LR-parsiranja koji se zasniva na traženju sinkronizacijskih znakova. \cite[str.~153-154]{udzbenik}

% 2009 - 1MI - 4 - (3)
\item
Navedite pet različitih vrsta sustava oznaka za opis sintaksnih pravila. \cite[str.~81-95]{udzbenik}

% 2009 - 1MI - 5 - (3)
\item
Opišite postupak sintaksne analize zasnovane na tablici Co-No. \cite[str.~83-84]{udzbenik}

% 2007 - 2MI - 1 - (3)
\item
Objasnite parsiranje od dna prema vrhu tehnikom \emph{Pomakni-Reduciraj}. 
Opišite tablice koje se koriste u parsiranju. \cite[str.~126-127]{udzbenik}

% 2007 - 1MI - 9 - (3)
\item
Opišite algoritme na kojima se zasnivaju postupci oporavka od pogreške kod sintaksne analize. \cite[str.~113,153]{udzbenik}

% 2004 - 2MI - 2 - (3)
\item
Objasnite parsiranje od dna prema vrhu metodom prednosti operatora, relaciju prednosti, akcije parsera i određivanje uzorka za zamjenu. \cite[str.~130-137]{udzbenik}

% 2003 - 2MI - 2 - (3)
\item
Objasnite razlike u ostvarenju parsera LR(0), SLR(1), LALR i LR(1) te navedite njihove prednosti i nedostatke. \cite[str.~138]{udzbenik}

% 2002 - 2MI - 3 - (3)
\item
Objasnite konstrukciju \(\varepsilon\)-NKA u postupku izgradnje SLR(1)-parsera. \cite[str.~147-149]{udzbenik}

% 2003 - 1MI - 7 - (3)
\item
Navedite zahtjeve koje mora ispuniti detekcija pogrešaka u sintaksnom analizatoru. \cite[str.~79]{udzbenik}

% 2002 - 1MI - 7 - (3)
\item
Objasnite sustav oznaka COBOL. \cite[str.~82]{udzbenik}

% 2001 - 2MI - 2 - (3)
\item
Opišite postupak oporavka od pogrešaka u sintaksnom analizatoru. \cite[str.~79-80]{udzbenik}

% 2001 - 2MI - 7 - (3)
% 2009 - P2MI - 3 - (3)
\item
Objasnite akcije parsera od dna prema vrhu koji koristi tehniku \emph{Pomakni-Pronađi}. Opišite proturječja koja se pojavljuju. \cite[str.~123-125]{udzbenik}

% 1999 - 2MI - 7 - (3)
\item 
Uklonite lijevu rekurziju iz sljedeće gramatike. \cite[str.~110-111]{udzbenik}

\begin{alltt}
<S> \(\to\) a<A>b<B>a | b<B>a<A>b
<A> \(\to\) <A>a<B>b | <B>a | a
<B> \(\to\) <A>b | b
\end{alltt} 

% 2007 - 1MI - 8 - (3)
\item
Uklonite lijevu rekurziju iz dane gramatike. \cite[str.~110-111]{udzbenik}

\begin{alltt}
<S> \(\to\) a<A>b<B>a | b<B>a<A>b
<A> \(\to\) <B>b | b
<B> \(\to\) <A>a | <B>a<A>b | a
\end{alltt} 

% 2001 - 2MI - 3 - (3)

\item
Uklonite lijevu rekurziju iz sljedeće gramatike. Odrediti je li dobivena gramatika LL(1). \cite[str.~110-111]{udzbenik}

\begin{alltt} 
<S> \(\to\) a<A>be<B>
<A> \(\to\) <B>cd | a<A> | \(\varepsilon\)
<B> \(\to\) <A>b<S>b | dc<S>a
\end{alltt} 

% 2010 - 2MI - 6 - (3)
\item
Na sljedeću gramatiku primijenite algoritam razrješavanja lijeve rekurzije: \cite[str.~110-111]{udzbenik}

\begin{alltt}
<S> \(\to\) <A>a<A>b
<A> \(\to\) <A>a | <A>b | c<A> | d<B> | f
<B> \(\to\) f
\end{alltt} 

% 2002 - 1MI - 5 - (3)

\item
Uklonite lijevu rekurziju iz zadane gramatike. \cite[str.~110-111]{udzbenik}

\begin{alltt}
<S> \(\to\) a<A><B> | b<B>
<A> \(\to\) <A>b | <B>ac | a
<B> \(\to\) <A>cb | bc
\end{alltt} 

% 2004 - 1MI - 8 - (3)
\item
Uklonite lijevu rekurziju iz zadane gramatike. Je li dobivena gramatika LL(1)? \cite[str.~110-111]{udzbenik}

\begin{alltt}
<S> \(\to\) <Z>ef | ef<S><X><Z>
<X> \(\to\) <Z>ee | fg<S>
<Z> \(\to\) <X>ff | fe<X>f | \(\varepsilon\)
\end{alltt}

% 1999 - 2MI - 1 - (3)
\item 
Za zadanu Q-gramatiku konstruirajte potisni automat.
Tijekom parsiranja ulaznog niza na vrhu stoga redom se pojavljuju sljedeći znakovi: \texttt{<S>}, \texttt{<A>}, \texttt{<B>}, \texttt{<B>}, \(\bigtriangledown\).
Koji su se ulazni znakovi sigurno nalazili u parsiranom nizu? \cite[str.~94-95]{udzbenik} \cite{auditorne}

\begin{alltt}
<S> \(\to\) a<A><B> | b<A><B>
<A> \(\to\) a<A> | b<A> | c | \(\varepsilon\)
<B> \(\to\) d | e<B> | f<A> | \(\varepsilon\)
\end{alltt} 
%aud/prim 9

% 1999 - 3MI - 8 - (3)
\item 
Napišite gramatiku na temelju koje je konstruiran navedeni potisni automat te odredite o kojoj se gramatici radi. \cite[str.~94-95]{udzbenik} \cite{auditorne}

\begin{tabular}{| c | c | c | c | }
\hline
   & a & b & $\perp$ \\ \hline
 S & 1 & 2 & 8 \\ \hline
 A & 3 & 4 & 8 \\ \hline
 B & 5 & 6 & 8 \\ \hline
 a & 7 & 8 & 8 \\ \hline
 b & 8 & 7 & 8 \\ \hline
 $\bigtriangledown$ & 8 & 8 & 9 \\ \hline
\end{tabular}

\noindent
1. zamijeni aBbA; pomakni \\
2. zamijeni bAaB; pomakni \\
3. zamijeni bA; pomakni \\
4. zamijeni B; zadrži \\
5. izvuci; zadrži \\
6. zamijeni aB; pomakni \\
7. izvuci; pomakni \\
8. odbaci \\
9. prihvati 
%aud/prim 10

% 2000 - 1MI - 5 - (3)
\item
Za zadanu gramatiku konstruirajte konačni potisni automat i izrazite ga pomoću tablice.
Prikažite rad potisnog automata na nizu \texttt{ebabc}. \cite[str.~94-95]{udzbenik} \cite{auditorne}

\begin{alltt}
<S> \(\to\) <A><B> | d<C>
<A> \(\to\) a<C><B> | \(\varepsilon\)
<B> \(\to\) b<C> | e<S>
<C> \(\to\) a<B> | c
\end{alltt} 
%aud/prim 9

% 2000 - 2MI - 8 - (3)
\item
Iz navedenog potisnog automata rekonstruirajte gramatiku. \cite[str.~94-95]{udzbenik} \cite{auditorne}

\begin{tabular}{| c | c | c | c | c | c | c | }
\hline
 & a & b & c & d & e & $\perp$ \\ \hline
$<$S$>$ & 1 & 9 & 9 & 1 & 1 & 9 \\ \hline
$<$A$>$ & 6 & 9 & 9 & 9 & 7 & 9 \\ \hline
$<$B$>$ & 2 & 9 & 9 & 3 & 2 & 9 \\ \hline
$<$C$>$ & 4 & 5 & 9 & 9 & 9 & 9 \\ \hline
$\bigtriangledown$ & 9 & 9 & 9 & 9 & 9 & 8 \\ \hline
\end{tabular}

\noindent
1: zadrži;	zamijeni ( d $<$A$>$ b $<$B$>$ ) \\
2: zadrži;	zamijeni ( d $<$A$>$ ) \\
3: pomakni;	zamijeni ( $<$C$>$ c $<$S$>$ ) \\
4: pomakni;	zamijeni ( $<$C$>$ ) \\
5: zadrži;	izvuci \\
6: pomakni;	zamijeni ( $<$A$>$ ) \\
7: pomakni;	izvuci \\
8: prihvati \\
9: odbaci \\
%aud/prim 10

% 2001 - 1MI - 5 - (3)
\item
Za zadanu gramatiku konstruirajte konačni potisni automat i izrazite ga pomoću tablice.
Prikažite rad potisnog automata na nizu \texttt{cdabed}. \cite[str.~94-95]{udzbenik} \cite{auditorne}

\begin{alltt}
<S> \(\to\) a<A>bc<B> | c<A>
<A> \(\to\) a<A> | b | d<A>e<C>
<B> \(\to\) b<B> | e<C>
<C> \(\to\) c | d | e
\end{alltt} 
%aud/prim 9

% 1999 - 2MI - 3 - (3)
\item 
Zadanu COBOL notaciju prevedite u BNF notaciju. \cite[str.~81-82]{udzbenik}

$\begin{Bmatrix}
\underline{A}\\ 
\underline{B}\\ 
\underline{C}D\\ 
\underline{C}E
\end{Bmatrix}
\begin{Bmatrix}
0\\ 
1\\ 
2
\end{Bmatrix}
...
\begin{Bmatrix}
\underline{A}\\ 
\underline{B}\\ 
D\underline{C}\\
E\underline{C}
\end{Bmatrix}$ 

% 2000 - 1MI - 9 - (3)
\item
Zadanu gramatiku u BNF notaciji prevedite u COBOL notaciju. \cite[str.~81-82]{udzbenik}

\begin{alltt}
<S> \(\to\) <D>CA | <D>CBA | <E>A
<D> \(\to\) <D>D | ABC
<E> \(\to\) <E>E | A
\end{alltt} 

% 2000 - 2MI - 5 - (3)
\item
Sljedeće pravilo zapisano u COBOL notaciji napišite u BNF notaciji: \cite[str.~81-82]{udzbenik}

\[
\underline{P}
\begin{Bmatrix}
\begin{bmatrix}
\underline{A}\\ 
\underline{N}
\end{bmatrix}

\begin{Bmatrix}
\underline{N}\\ 
K\underline{B}
\end{Bmatrix}\\ 
\\
T\underline{NK}
\end{Bmatrix}
...
\] 

% 2001 - 1MI - 1 - (3)
\item
Navedeni izraz zapisan u COBOL notaciji pretvorite u BNF notaciju. \cite[str.~81-82]{udzbenik}

\(A [B] ... [A [B] ...]...C[C]...\)

% 2002 - 2MI - 2 - (3)
\item
Zadani BNF zapis pretvorite u COBOL zapis. \cite[str.~81-82]{udzbenik}

\begin{alltt}
<if> \(\to\) IF <izraz> THEN <blok>
<if> \(\to\) IF <izraz> THEN <blok> ELSE <blok>
<blok> \(\to\) N
<blok> \(\to\) BEGIN <naredbe> END
<naredbe> \(\to\) N
<naredbe> \(\to\) N <naredbe>
<izraz> \(\to\) A=B
<izraz> \(\to\) A<>B
<izraz> \(\to\) A<B
<izraz> \(\to\) A<=B
<izraz> \(\to\) A>B
<izraz> \(\to\) A>=B
\end{alltt} 

% 2003 - 1MI - 6 - (3)
\item
Zadanu BNF notaciju pretvorite u COBOL notaciju. \cite[str.~81-82]{udzbenik}

\begin{alltt}
<S> \(\to\) a<A> | b<B> | c<C>
<A> \(\to\) ab<A> | bac<A> | c
<B> \(\to\) a | bc<B> | c<C>ca
<C> \(\to\) ab | ba
\end{alltt}

% 2004 - 1MI - 5 - (3)
\item
Zadani COBOL zapis pretvorite u BNF zapis. \cite[str.~81-82]{udzbenik}

\[
\begin{Bmatrix}
\underline{C} \\ 
\underline{B} \\ 
\underline{L}
\end{Bmatrix}
...
[\underline{NOT}]
\begin{Bmatrix}
\underline{2}\\ 
\underline{TO}...
\end{Bmatrix}
\underline{B}
[\underline{NOT}]
...
\] 

% 2007 - 1MI - 5 - (3)
\item
Zadani COBOL zapis pretvorite u istovjetni BNF zapis. \cite[str.~81-82]{udzbenik}

\[
IF \ A \begin{Bmatrix}
= \\ 
!= \\ 
< \\ 
>
\end{Bmatrix}
B \ THEN
\begin{Bmatrix}
N\\ 
BEGIN \ N \ [N] \ END
\end{Bmatrix}
\begin{bmatrix}
ELSE \begin{Bmatrix}
N\\ 
BEGIN \ N \ [N] \ END
\end{Bmatrix}
\end{bmatrix}
\] 

% 1999 - 1MI - 7 - (3)
\item
Sljedeću gramatiku zapisanu u BNF notaciji prevedite u COBOL notaciju: \cite[str.~81-82]{udzbenik}

\begin{alltt}
<realni_broj> \(\to\) <brojka> | <brojka> <rb1> 
<rb1> \(\to\) <brojka> | <brojka> <rb1> | . <rb2> | <rb3> 
<rb2> \(\to\) <brojka> | <brojka> <rb2> | <rb3> 
<rb3> \(\to\) E <rb4> | e <rb4> 
<rb4> \(\to\) + <rb5> | - <rb5> | <rb5> 
<rb5> \(\to\) <brojka> | <brojka> <rb5> 
<brojka> \(\to\) 0 | 1 | 2 | 3 | 4 | 5 | 6 | 7 | 8 | 9 
\end{alltt} 

% 2010 - P3MI-L - 6 - (3)
% 2010 - P2MI-L - 6 - (3)
% 2010 - P1MI-L - 6 - (3)
\item
Primjenom sustava oznaka BNF opišite konstante s pomičnom točkom.
Na primjer, \texttt{-2.3e10}, \texttt{0.02} i \texttt{.2e-15} ispravno su zapisane konstante s pomičnom točkom. \cite[str.~81]{udzbenik}

% 2009 - P1MI - 6 - (3)
% 2009 - P2MI - 5 - (3)
\item
Primjenom sustava oznaka COBOL opišite konstante s pomičnim zarezom.
Npr.~konstante oblika \texttt{-3.3e10}, \texttt{-02.0}. \cite[str.~82]{udzbenik}

% 2010 - P3MI - 6 - (3)
% 2010 - P2MI - 6 - (3)
% 2010 - P1MI - 6 - (3)
% AV - 3 - (3)
\item
BNF sustavom oznaka opišite BNF sustav oznaka.
Za varijable koristite uniformni znak \texttt{VARIJABLA}, a za konstante uniformni znak \texttt{KONSTANTA}. \cite[str.~81]{udzbenik}

% 2009 - 3MI - 7 - (3)
\item
BNF sustavom oznaka opišite niz naredbi definicije varijabli tipa \texttt{int} u programskom jeziku C.
Dopuštena imena varijabli su \texttt{a}, \texttt{b} i \texttt{c}.
Početne su vrijednosti varijabli opcionalne, a mogu biti isključivo dekadski brojevi.
Primjer jednog niza naredbi je: \cite[str.~81]{udzbenik}

\begin{alltt}
int a, b=261;
int c;
int a=123;
\end{alltt}

% 2002 - 3MI - 1 - (3)
\item
Zadana je Co-No tablica.
Odredite rezultat izvođenja zadanog programa. \cite[str.~83-84]{udzbenik} \cite{auditorne}

\texttt{,7\(\to\)x,12\(\to\)y,x*y\(\to\)x,3\(\to\)z,x/z\(\to\)y,x-y/2\(\to\)x,x*2/y+z\(\to\)x,x*6-y+z*2-9*x\(\to\)y,x+y/2\(\to\)z,}

\begin{tabular}{ | c | c | c | c | c | c | c | } \hline
 & , & \(\to\) & * & / & + & - \\ \hline
, & greška & dohvati & dohvati & dohvati & dohvati & dohvati  \\ \hline
\(\to\) & spremi & greška & greška & greška & greška & greška \\ \hline
* & greška & pomnoži & pomnoži & pomnoži & pomnoži & pomnoži \\ \hline
/ & greška & podijeli & podijeli & podijeli & podijeli & podijeli \\ \hline
+ & greška & zbroji & zbroji & zbroji & zbroji & zbroji  \\ \hline
- & greška & oduzmi & oduzmi & oduzmi & oduzmi & oduzmi \\ \hline
\end{tabular}
%aud/prim 7

% 2003 - 1MI - 5 - (3)
\item
Primijenite zadanu Co-No tablicu u parsiranju sljedećeg niza: \texttt{,3+5*2\(\to\)R,6+2*3<R>18,}
Odredite vrijednost koja se na kraju rada parsera nalazi u akumulatoru. \cite[str.~83-84]{udzbenik} \cite{auditorne}

\begin{tabular}{ | c | c | c | c | c | c | c | } \hline
 & , & \(\to\) & $<$ & $>$ & + & * \\ \hline
, &  & \#1 & \#1 & \#1 & \#1 & \#1  \\ \hline
\(\to\) & \#2 & \#2 & \#2 & \#2 & \#2 & \#2 \\ \hline
$<$ &  & \#3 &  & \#3 & \#3 & \#3 \\ \hline
$>$ &  & \#4 &  & \#4 & \#4 & \#4 \\ \hline
+ &  & \#5 & \#5 & \#5 & \#5 & \#5  \\ \hline
* &  & \#6 & \#6 & \#6 & \#6 & \#6 \\ \hline
\end{tabular}

\#1:	dohvati operand u akumulator \\
\#2:	spremi vrijednost akumulatora u operand \\
\#3:	ako je operand manji od akumulatora, stavi ga u akumulator \\
\#4:	ako je operand veći od akumulatora, stavi ga u akumulator \\
\#5:	pribroji operand akumulatoru \\
\#6:	pomnoži akumulator s operandom
%aud/prim 7

% AV - 5 - (3)
\item
Zadanom Co-No tablicom parsirajte dva niza naredbi.
Odredite prihvaća li se zadani niz zadanom Co-No tablicom, napišite generirani niz naredbi ciljnog programa i odredite vrijednosti varijabli \texttt{a}, \texttt{b} i \texttt{c} nakon izvođenja ciljnog programa.\\
a) ; 4 \(\to\) a ; 5 \(\to\) b ; a + b * 10 \(\to\) c ; \\
b) ; 5 \(\to\) c ; 3 \(\to\) d ; c * d ; c / d \(\to\) a ;

Tablica sadrži akcije generatora ciljnog programa za stogovni stroj.
Akcija \texttt{PUSH} stavlja na vrh stoga zadanu vrijednost ili vrijednost zadane verijable.
Akcija \texttt{POP} skida podatak s vrha stoga i sprema ga u zadanu varijablu.
Akcije \texttt{ADD}, \texttt{SUB}, \texttt{MUL} i \texttt{DIV} skidaju dva podatka s vrha stoga, izvode operaciju i stavljaju rezultat na vrh stoga. Akcija -- označava grešku u ulaznom nizu. \cite[str.~83-84]{udzbenik} \cite{auditorne}

\begin{tabular}{c c | c | c | c | c | c | c | }
\cline{3-8}
& & \multicolumn{6}{|c|}{Desni operator} \\ \cline{3-8}
& & ; & + & - & * & / & \(\to\) \\ \cline{1-8}
\multicolumn{1}{|c|}{} & ; & -- & \#1; & \#1; & \#1;  & \#1; & \#1;     \\ \cline{2-8}
\multicolumn{1}{|c|}{} & + & -- & \#1; \#2; & \#1; \#2; & \#1; \#2; & \#1; \#2; & \#1; \#2;    \\ \cline{2-8}
\multicolumn{1}{|c|}{Lijevi} & - & -- & \#1; \#3; & \#1; \#3; & \#1; \#3; & \#1; \#3; & \#1; \#3; \\ \cline{2-8}
\multicolumn{1}{|c|}{operator} & * & -- & \#1; \#4; & \#1; \#4; & \#1; \#4; & \#1; \#4; & \#1; \#4; \\ \cline{2-8}
\multicolumn{1}{|c|}{} & / & -- & \#1; \#5; & \#1; \#5; & \#1; \#5; & \#1; \#5; & \#1; \#5; \\ \cline{2-8}
\multicolumn{1}{|c|}{} & \(\to\) & \#6; & -- & -- & -- & -- & -- \\ \cline{1-8}
\end{tabular}

pri čemu su akcije u tablici:\\
\#1: PUSH\\
\#2: ADD\\
\#3: SUB\\
\#4: MUL\\
\#5: DIV\\
\#6: POP\\ 
%aud/prim 7

% 1999 - 1MI - 5 - (3)
\item
Primijenite zadanu Co-No tablicu u parsiranju sljedećeg niza: \texttt{,x+y+z>w*x\(\rightarrow\)u\(\rightarrow\)v+z\(\rightarrow\)R,}

\begin{tabular}{c c | c | c | c | c | c | }
\cline{3-7}
& & \multicolumn{5}{|c|}{Desni operator} \\ \cline{3-7}
& & , & $\to$ & $>$ & + & * \\ \cline{1-7}
\multicolumn{1}{|c|}{} & , &  & \#1 & \#1 & \#1 & \#1  \\ \cline{2-7}
\multicolumn{1}{|c|}{} & $\to$ & \#2 & \#2 & & &  \\ \cline{2-7}
\multicolumn{1}{|c|}{Lijevi operator} & $>$ & & \#3 &  & \#3 & \#3 \\ \cline{2-7}
\multicolumn{1}{|c|}{} & + &  & \#4 & \#4 & \#4 & \#4 \\ \cline{2-7}
\multicolumn{1}{|c|}{} & * &  & \#5 & \#5 & \#5 & \#5 \\ \cline{1-7}
\end{tabular}

pri čemu su akcije u tablici:\\
\#1: dohvati operand u akumulator \\
\#2: spremi vrijednost akumulatora \\
\#3: ako je operand veći od akumulatora, stavi ga u akumulator \\
\#4: pribroji operand akumulatoru \\
\#5: pomnoži akumulator s operandom \cite[str.~83-84]{udzbenik} \cite{auditorne}
%aud/prim 7

% 2000 - 1MI - 7 - (3)
\item
Zadana je Co-No tablica:

\begin{tabular}{ c c | c | c | c | c | c | c | c | }
\cline{3-8}
& & \multicolumn{6}{|c|}{Desni operator} \\ \cline{3-8}
& & , & \(\to\) & + & - & * & / \\ \cline{1-8}
\multicolumn{1}{|c|}{} & , & greška & spremi & greška & greška & greška & greška \\ \cline{2-8}
\multicolumn{1}{|c|}{} & \(\to\) & dohvati & greška & zbroji & oduzmi & moži & dijeli \\ \cline{2-8}
\multicolumn{1}{|c|}{} & + & dohvati & greška & zbroji & oduzmi & moži & dijeli \\ \cline{2-8}
\multicolumn{1}{|c|}{Lijevi operator} & - & dohvati & greška & zbroji & oduzmi & moži & dijeli \\ \cline{2-8}
\multicolumn{1}{|c|}{} & * & dohvati & greška & zbroji & oduzmi & moži & dijeli \\ \cline{2-8}
\multicolumn{1}{|c|}{} & / & dohvati & greška & zbroji & oduzmi & moži & dijeli \\ \hline
\end{tabular}

Odredite vrijednosti varijabli \texttt{a}, \texttt{b} i \texttt{c} nakon izvođenja sljedećeg programa:

\texttt{,1\(\to\)a,2\(\to\)b,3\(\to\)c,a*b+c\(\to\)c,c-1/a\(\to\)b,c+b*a\(\to\)a,a*7-b+3*a-c*4+7\(\to\)c,} \cite[str.~83-84]{udzbenik} \cite{auditorne}
%aud/prim 7

% 2001 - 1MI - 8 - (3)
\item
Napišite Co-No tablicu za parsiranje aritmetičkih nizova koji sadrže operacije \texttt{+}, \texttt{*}, \texttt{<} i \texttt{>}.
Kao separator koristi se \texttt{,} (zarez), a za pohranu vrijednosti u akumulator koristi se znak \texttt{\(\to\)}.
Operator \texttt{<} uspoređuje sadržaj akumulatora s operandom i u akumulator stavlja manji od njih.
Analogno vrijedi i za operator \texttt{>}, s tim da se u akumulator stavlja veći od njih.
Na primjer, za \texttt{2<3} u akumulatoru ostaje \texttt{2}, za \texttt{4<3} u akumulator se stavlja \texttt{3}, za \texttt{2>3} u akumulator se stavlja \texttt{3} te za \texttt{4>3} u akumulatoru ostaje \texttt{4}.
Odredite vrijednost varijable \texttt{b} nakon parsiranja sljedećeg programa: \texttt{,1+2\(\to\)a,2*2<a>2\(\to\)b,}  \cite[str.~83-84]{udzbenik} \cite{auditorne}
%aud/prim 7

% 2004 - 3MI - 5 - (3)
\item
Izgradite tablicu relacija prednosti za zadanu operatorsku gramatiku. \cite[str.~133-135]{udzbenik}

\begin{alltt}
<S> \(\to\) <A>a<S> | <B>
<A> \(\to\) <B>c<D> | <D>d
<B> \(\to\) b
<D> \(\to\) c<S>a
\end{alltt} 

% 2007 - 3MI - 1 - (3)
% 2000 - 2MI - 6 - (3)
\item
Na temelju zadane operatorske gramatike izgradite tablicu relacija prednosti. Prikažite parsiranje niza \( (a \vee a \wedge \lnot a) \vee \lnot a \wedge a \) pomoću izgrađene tablice. \cite[str.~133-135]{udzbenik}

\begin{alltt}
<E> \(\to\) <E> \(\wedge\) <T> | <T>
<T> \(\to\) <T> \(\vee\) <P> | <P>
<P> \(\to\) <N> | \(\lnot\)<N>
<N> \(\to\) ( <E> )
<N> \(\to\) a
\end{alltt} 

% 2010 - 2MI - 8 - (3)
% 2009 - 2MI - 6 - (3)
\item
Binarni operator \texttt{\#} lijevo je asocijativan i manje prednosti od binarnog operatora \texttt{\$}.
Operator \texttt{\$} desno je asocijativan.
U izrazima je dopušteno korištenje okruglih zagrada, a varijable su označene završnim znakom \texttt{a}.
Pripremite tablicu prednosti za parser zasnovan na prednosti operatora koji će parsirati opisane nizove. \cite[str.~135-136]{udzbenik}

% 2001 - 2MI - 5 - (3)
\item
Odredite PRIMIJENI skupove svih produkcija u zadanoj gramatici.
Odredite je li gramatika LL(1) i obrazložite. \cite[str.~100]{udzbenik}

\begin{alltt} 
<S> \(\to\) a<S>b<S> | c<D><A> | \(\varepsilon\)
<A> \(\to\) <B>c | ba<B>
<B> \(\to\) a<D> | de<C> | e<C>
<C> \(\to\) <D>f | a<B>c
<D> \(\to\) e<D> | \(\varepsilon\)
\end{alltt} 

% 2010 - 3MI - 7 - (3)
\item
Odredite PRIMIJENI skupove svih produkcija u zadanoj gramatici.
Je li zadana gramatika tipa LL(1)? \cite[str.~100]{udzbenik}

\begin{alltt}
<S> \(\to\) a<S> | b<X>
<X> \(\to\) <Y> | a
<Y> \(\to\) <S>a<X> | c | \(\varepsilon\)
\end{alltt} 

% 2009 - P2MI-L - 6 - (3)
\item
Odredite PRIMIJENI skupove svih produkcija u zadanoj gramatici.
Je li zadana gramatika tipa LL(1)?
Obrazložite odgovor. \cite[str.~100]{udzbenik}

\begin{alltt}
<S> \(\to\) a<S>c<S><Z> | b<S>c<Y>c<S>
<Y> \(\to\) a<Y><Z>c | cbb<Z>
<Z> \(\to\) <S>a<Z> | c<Z>c<Y> | \(\varepsilon\)
\end{alltt} 

% 2009 - P2MI - 7 - (3)
% 2002 - 1MI - 8 - (3)
\item
Odredite PRIMIJENI skupove svih produkcija u zadanoj gramatici.
Je li zadana gramatika tipa LL(1)?
Obrazložite odgovor. \cite[str.~100]{udzbenik}

\begin{alltt}
<S> \(\to\) <C>c<E> | b<A>b<B> | ded<A>a | \(\varepsilon\)
<A> \(\to\) c<B>f | f<B>d | \(\varepsilon\)
<B> \(\to\) <C>b | c<D> | \(\varepsilon\)
<C> \(\to\) ad<C>d | e<E>c<D><A>d | \(\varepsilon\)
<D> \(\to\) a<S>fe | b<B>de | \(\varepsilon\)
<E> \(\to\) da<C> | e<A>e<S>fa
\end{alltt} 

% 2003 - 2MI - 3 - (3)
\item
Konstruirajte potisni automat za zadanu LL(1)-gramatiku te odredite koji su mogući ulazni nizovi ako su se tijekom parsiranja na vrhu stoga redom pojavili nezavršni znakovi \texttt{<S>} \texttt{<S>} \texttt{<A>} \texttt{<A>} \texttt{<B>} \texttt{<B>} (namjerno nije navedeno pojavljivanje završnih znakova na vrhu stoga). \cite[str.~99]{udzbenik}

\begin{alltt}
<S> \(\to\) a<S>c | b<A>d
<A> \(\to\) <B> | b<A>d
<B> \(\to\) e<B> | \(\varepsilon\)
\end{alltt} 

% 2003 - 1MI - 8 - (3)
% 1999 - 2MI - 5 - (3)
\item
Odredite PRIMIJENI skupove za produkcije zadane gramatike (nije potrebno koristiti algoritam od 12 koraka).
Odredite je li gramatika LL(1) i obrazložite odgovor. \cite[str.~100]{udzbenik}

\begin{alltt}
<S> \(\to\) a<A><S>b<C><D> | \(\varepsilon\)
<A> \(\to\) c<A><S>e | \(\varepsilon\)
<B> \(\to\) a<S>fc | cba<B> | \(\varepsilon\)
<C> \(\to\) ab<C>d | c<C>e | \(\varepsilon\)
<D> \(\to\) <A>a<B>f | d<B><D> | \(\varepsilon\)
\end{alltt} 

% 2000 - 2MI - 3 - (3)
\item
Odredite PRIMIJENI skupove za produkcije dane konteksno neovisne gramatike čiji je početni nezavršni znak \texttt{<A>}.
Odredite kojeg tipa je ta gramatika i navedite uvjete koje ispunjavaju gramatike tog tipa. \cite[str.~85-115]{udzbenik}

\begin{alltt}
<A> \(\to\) a<E> | c | \(\varepsilon\)
<B> \(\to\) b<C>b<A> | d
<C> \(\to\) a<B>e | b | d<A>b<D>
<D> \(\to\) c<C>f | \(\varepsilon\)
<E> \(\to\) a<D>a | c<B>f
\end{alltt} 

% AV - 4 - (3)
\item 
Zadanu gramatiku pretvorite u S-gramatiku. \cite[str.~85]{udzbenik} \cite{auditorne}

\begin{alltt}
<S> \(\to\) a<A>b | b<A>c | c<B>a
<A> \(\to\) d | \(\varepsilon\)
<B> \(\to\) <B>c | g
\end{alltt} 
%aud/prim 6

% 2002 - 1MI - 6 - (3)
\item
Zadanu gramatiku pretvorite u S-gramatiku. \cite[str.~85]{udzbenik} \cite{auditorne}

\begin{alltt}
<S> \(\to\) <C> | a<A>c | b<B>a
<A> \(\to\) ab<A> | bac<A> | \(\varepsilon\)
<B> \(\to\) <C>c | bc<B> | \(\varepsilon\)
<C> \(\to\) cab | cba
\end{alltt} 
%aud/prim 6

% 2009 - P1MI - 10 - (3)
% 2009 - P2MI-E - 3 - (3)
% 2010 - P3MI - 10 - (3)
\item
Zadanu gramatiku pretvorite u S-gramatiku koja generira isti jezik. \cite[str.~85]{udzbenik} \cite{auditorne}

\begin{alltt}
<S> \(\to\) <C> | a<A>c | b<B>a
<A> \(\to\) ab<A> | bac<A> | \(\varepsilon\)
<B> \(\to\) <C>c | bc<B> | \(\varepsilon\)
<C> \(\to\) cab | cba
\end{alltt} 
%aud/prim 6

% 2009 - 3MI - 8 - (3)
\item
Pretvorite zadanu gramatiku u S-gramatiku.
Provedite i postupak pojednostavljenja dobivene gramatike. \cite[str.~85]{udzbenik} \cite{auditorne}

\begin{alltt}
<S> \(\to\) ab<A>
<A> \(\to\) <B>bc | cb<A>
<B> \(\to\) a<S>a | cc<B>b | \(\varepsilon\)
\end{alltt} 
%aud/prim 6

% 2010 - P2MI-L - 9 - (3)
\item
Za zadanu kontekstno neovisnu gramatiku konstruirajte istovjetnu gramatiku koja nema \(\varepsilon\)-produkcija i u kojoj desna strana svake produkcije započinje završnim znakom. \cite[str.~85]{udzbenik} \cite{auditorne}

\begin{alltt}
<A> \(\to\) a<B>a | ac<B>c
<B> \(\to\) <B>a | a<A> | \(\varepsilon\)
\end{alltt} 
%aud/prim 6

% 2007 - 1MI - 10 - (3)
\item
Zadanu gramatiku pretvorite u Q-gramatiku i pokažite da dobivena gramatika zadovoljava pravila Q-gramatike. \cite[str.~91-95]{udzbenik}

\begin{alltt}
<S> \(\to\) <A>c<D> | c
<A> \(\to\) <B>a | b<D>c
<B> \(\to\) <C>e | ef<A>
<C> \(\to\) <B> | a
<D> \(\to\) a<D> | b
\end{alltt} 

% 2004 - 1MI - 10 - (3)
\item
Zadanu Q-gramatiku pretvorite u S-gramatiku. \cite[str.~85-95]{udzbenik}

\begin{alltt}
<S> \(\to\) a<A><C>c | b<B><A>a
<A> \(\to\) d<C> | \(\varepsilon\)
<B> \(\to\) f | \(\varepsilon\)
<C> \(\to\) e<B>
\end{alltt} 

% 2009 - P3MI - 3 - (3)
% 2003 - 2MI - 5 - (3)
\item
Zadanu gramatiku pretvorite u LL(1)-gramatiku koja generira isti jezik: \cite[str.~107-111]{udzbenik}

\begin{alltt}
<S> \(\to\) <S>ca | a<A>c | ab<S>b
<A> \(\to\) c<A>c | d<B>ac | db<A>c
<B> \(\to\) baa<B> | bbc<B> | d
\end{alltt} 

% 1999 - 3MI - 4 - (3)
\item 
Zadanu gramatiku prevedite u Q-gramatiku, uz pretpostavku da su \texttt{kon} i \texttt{var} završni znakovi. \cite[str.~91-98]{udzbenik}

\begin{alltt}
<E> \(\to\) <E>+<T> | <T>
<T> \(\to\) <P> | <T>*<P>
<P> \(\to\) (<E>) | kon | var
\end{alltt} 

% AV - 9 - (3)
\item
Za zadanu gramatiku izgradite parser zasnovan na tehnici parsiranja \emph{Pomakni-Pronađi}. \cite[str.~121-123]{udzbenik}

\begin{alltt}
<S> \(\to\) b<A> | p<A>m<C>
<A> \(\to\) d<S>a | e
<C> \(\to\) d<A>
\end{alltt} 
%aud/prim 11

% 2009 - 1MI - 10 - (3)
% 2010 - P1MI-L - 10 - (3)
\item
Odredite tip zadane gramatike i konstruirajte potisni automat parsera od vrha prema dnu.  \cite[str.~84-100]{udzbenik}

\begin{alltt}
<S> \(\to\) a<A>b<B> | b<B><A>
<A> \(\to\) a<C>
<B> \(\to\) b<A> | \(\varepsilon\)
<C> \(\to\) cc
\end{alltt}

% 2009 - 2MI - 8 - (3)
\item
Za zadanu gramatiku izgradite parser zasnovan na tehnici \emph{Pomakni-Pronađi}. \cite[str.~121-123]{udzbenik} \cite{auditorne}

\begin{alltt}
<S> \(\to\) ab<A>
<A> \(\to\) ab<B> | bb
<B> \(\to\) c
\end{alltt} 
%aud/prim 11

% 2007 - 2MI - 3 -(3)
\item
Za zadanu gramatiku konstruirajte parser zasnovan na tehnici \emph{Pomakni-Pronađi}. \cite[str.~121-123]{udzbenik} \cite{auditorne}

\begin{alltt}
<S> \(\to\) aba<S>q<W> | baf<Y>
<W> \(\to\) cc<S> | q
<Y> \(\to\) caba<W> | ffa<S>q<W>
\end{alltt} 
%aud/prim 11

% 2003 - 2MI - 9 - (3)
\item 
Za zadanu gramatiku konstruirajte parser od dna prema vrhu metodom \emph{Pomakni-Pronađi}. \cite[str.~121-123]{udzbenik} \cite{auditorne}

\begin{alltt}
<S> \(\to\) a<S><A><B> | b<B>
<A> \(\to\) bad | c<A><B>e
<B> \(\to\) ab | d<A>e
\end{alltt} 
%aud/prim 11

% 2010 - P2MI - 7 - (3)
\item
Za zadanu Q-gramatiku konstruirajte potisni automat. \cite[str.~94-95]{udzbenik} \cite{auditorne}

\begin{alltt}
<S> \(\to\) ffg<A><B>
<A> \(\to\) f<C>ae<A><B> | \(\varepsilon\)
<B> \(\to\) ga<A>
<C> \(\to\) ee | g<S>
\end{alltt}
%aud/prim 9

% 2007 - 1MI - 3 - (3)
\item
Konstruirajte potisni automat za zadanu gramatiku. \cite[str.~94-95]{udzbenik}

\begin{alltt}
<S> \(\to\) a<B><A>b | b<A><B>c
<A> \(\to\) ba<B>ba | bb<B>bb | c<B>c
<B> \(\to\) a<B>a | \(\varepsilon\)
\end{alltt} 

% 2003 - 1MI - 10 - (3)
\item
Konstruirajte potisni automat koji prihvaća nizove koje generira zadana gramatika. \cite[str.~98-99]{udzbenik}

\begin{alltt}
<S> \(\to\) a<A>b<B>c | b<B>a<A>a
<A> \(\to\) <A>c<B>a | <B>a | a
<B> \(\to\) <B>b | b
\end{alltt} 

% 2004 - 1MI - 6 - (3)
% AV - 7 - (3)
\item
Konstruirajte potisni automat za zadanu Q-gramatiku. \cite[str.~94-95]{udzbenik} \cite{auditorne}

\begin{alltt}
<S> \(\to\) a<A><B>c | c<B><A>b
<A> \(\to\) a<A> | \(\varepsilon\)
<B> \(\to\) b<B> | c
\end{alltt} 
%aud/prim 9

% 2001 - 3MI - 3 - (3)
\item
Konstruirajte jednostavni potisni automat za navedenu LL(1)-gramatiku.
Automat prikažite u tabličnom obliku. \cite[str.~99]{udzbenik}

\begin{alltt}
<S> \(\to\) <B><B>b | cba<A> | \(\varepsilon\)
<A> \(\to\) ca<B> | dc<A>a
<B> \(\to\) a<A><S>d | b<S>d
\end{alltt} 

% 2009 - P1MI - 9 - (3)
% 2010 - P3MI-L - 7 - (3)
% 2010 - P2MI-L - 7 - (3)
\item
Prikažite stanje stoga i položaj glave za čitanje ulazne trake tijekom parsiranja niza abca primjenom zadanog potisnog automata.
Koje je akcije potisnog automata potrebno pridružiti ćeliji s oznakom \texttt{X} da bi niz \texttt{abca} bio prihvatljiv? 
Obrazložite odgovor. \cite[str.~90]{udzbenik}

\begin{tabular}{|c|c|c|c|c|} \hline
 & a & b & c & $\perp$ \\ \hline
S & \#1 & \#2 & - & - \\ \hline
A & - & - & \#3 & - \\ \hline
B & X & - & - & - \\ \hline
b & - & \#5 & - & - \\ \hline
$\bigtriangledown$ & - & - & - & \#6 \\ \hline
\end{tabular}

\#1:	ZAMIJENI(Ab); POMAKNI; \\
\#2:	ZAMIJENI(B); POMAKNI; \\
\#3:	ZAMIJENI(B); POMAKNI; \\
\#5:	IZVUCI; POMAKNI; \\
\#6:	PRIHVATI; \\
-:	ODBACI; \\

% 2010 - 1MI - 10 - (3)
\item
Neka je zadana sljedeća S-gramatika:

\begin{alltt}
<S> \(\to\) ab<R> | b<R>b<S>
<R> \(\to\) a | b<R>
\end{alltt}

te neka je zadana sljedeća tablica prijelaza potisnog automata:

\begin{tabular}{|c|c|c|c|} \hline
 & a & b & $\perp$ \\ \hline
$<$S$>$ & (1) & (2) & Prihvati \\ \hline
$<$R$>$ & (3) & (4) & - \\ \hline
b & - & (3) & - \\ \hline
$\bigtriangledown$ & - & - & - \\ \hline
\end{tabular}
\\
(1) Zamijeni($<$R$>$ b a); Pomakni; \\
(2) Zamijeni($<$S$>$ b $<$R$>$ b); Pomakni; \\
(3) Izvuci; Pomakni; \\
(4) Zamijeni($<$R$>$); \\
(-) Odbaci;

Utvrdite prihvaća li potisni automat s navedenom tablicom prijelaza jezik zadan navedenom S-gramatikom.
Ako ne, navedite potrebne izmjene u tablici kako bi potisni automat prihvaćao jezik. \cite[str.~85-89]{udzbenik}

% 2002 - 1MI - 10 - (3)
\item
Odredite produkcije i vrstu gramatike na temelju koje je konstruiran sljedeći potisni automat. \cite[str.~85-99]{udzbenik} \cite{auditorne}

\begin{tabular}{| c | c | c | c | c | }
\hline
 & a & b & c & $\perp$ \\ \hline
S & 1 & 2 & 3 & 8 \\ \hline
A & 2 & 4 & 8 & 4 \\ \hline
B & 5 & 4 & 6 & 8 \\ \hline
b & 8 & 7 & 8 & 8 \\ \hline
$\bigtriangledown$ & 8 & 8 & 8 & 9 \\ \hline
\end{tabular}

1: zamijeni (cA);	pomakni \\
2: zamijeni (S);	pomakni \\
3: zamijeni (bB);	pomakni \\
4: izvuci;	zadrži \\
5: zamijeni (AbB);	pomakni \\
6: zamijeni (SS);	pomakni \\
7: izvuci;	pomakni \\
8: odbaci \\
9: prihvati \\
%aud/prim 10

% 2001 - 2MI - 8 - (3)
\item
Odredite gramatiku na temelju koje je konstruiran sljedeći potisni automat. \cite[str.~85-99]{udzbenik} \cite{auditorne}

\begin{tabular}{| c | c | c | c | c | c | } \hline
 & 1 & 2 & 3 & 4 & $\perp$ \\ \hline
$<$S$>$ & 1 & 2 & 7 & 1 & 2 \\ \hline
$<$A$>$ & 3 & 7 & 7 & 4 & 7 \\ \hline
$<$B$>$ & 7 & 5 & 7 & 7 & 7 \\ \hline
$<$C$>$ & 7 & 7 & 7 & 6 & 7 \\ \hline
$\bigtriangledown$ & 7 & 7 & 7 & 7 & 8 \\ \hline
\end{tabular}

1: zadrži; zamijeni (2B1A) \\
2: zadrži; izvuci \\
3: pomakni; zamijeni (A) \\
4: zadrži; zamijeni (C) \\
5: pomakni; izvuci \\
6: pomakni; zamijeni (2S) \\
7: odbaci \\
8: prihvati \\
%aud/prim 10

% 2004 - 2MI - 5 - (3)
\item
Napišite gramatiku na temelju koje je nastao zadani potisni automat i odredite tip gramatike. \cite[str.~85-99]{udzbenik} \cite{auditorne}

\begin{tabular}{ | c | c | c | c | c | c | c | } \hline
 & a & b & c & d & e & $\perp$ \\ \hline
$<$S$>$ & 9 & 9 & 9 & 9 & 4 & 9 \\ \hline
$<$A$>$ & 9 & 9 & 9 & 9 & 5 & 9 \\ \hline
$<$B$>$ & 9 & 9 & 9 & 9 & 6 & 9 \\ \hline
$<$C$>$ & 1 & 1 & 9 & 9 & 7 & 7 \\ \hline
$<$D$>$ & 9 & 9 & 9 & 9 & 3 & 9 \\ \hline
$<$E$>$ & 7 & 7 & 2 & 2 & 7 & 7 \\ \hline
e & 9 & 9 & 9 & 9 & 8 & 9  \\ \hline
$\bigtriangledown$ & 9 & 9 & 9 & 9 & 9 & 10 \\ \hline
\end{tabular}
\\

 1: zamijeni ($<$C$>$$<$B$>$); pomakni \\
 2: zamijeni ($<$E$>$$<$D$>$); pomakni \\
 3: zamijeni (e$<$A$>$);   pomakni \\
 4: zamijeni ($<$A$>$);    zadrži \\
 5: zamijeni ($<$C$>$$<$B$>$); zadrži \\
 6: zamijeni ($<$E$>$$<$D$>$); zadrži \\
 7: izvuci;            zadrži \\
 8: izvuci;            pomakni \\
 9: odbaci \\
10: prihvati \\
%aud/prim 10

% 1999 - 1MI - 10 - (3)
% 2010 - 2MI - 7 - (3)
\item
Rekonstruirajte gramatiku iz koje je nastao sljedeći potisni automat \cite[str.~85-99]{udzbenik} \cite{auditorne}

\begin{tabular}{|c|c|c|c|c|c|c|} \hline
& ( & ) & + & * & konst & $\perp$ \\ \hline
$<$E$>$ & 1 & 2 & 2 & 2 & 3 & 2 \\ \hline
$<$T$>$ & 2 & 2 & 4 & 5 & 2 & 5 \\ \hline
$<$P$>$ & 2 & 6 & 2 & 4 & 2 & 6 \\ \hline
) & 2 & 7 & 2 & 2 & 2 & 2 \\ \hline
$\bigtriangledown$ & 2 & 2 & 2 & 2 & 2 & 8 \\ \hline
\end{tabular}

1: Zamijeni($<$T$>$)$<$E$>$); Pomakni; \\
2: Odbaci; \\
3: Zamijeni($<$T$>$); Pomakni; \\
4: Zamijeni($<$E$>$); Pomakni; \\
5: Zamijeni($<$P$>$); Zadrži; \\
6: Izvuci; Zadrži; \\
7: Izvuci; Pomakni; \\
8: Prihvati; \\
%aud/prim 10

% AV - 8 - (3)
\item
Odredite produkcije gramatike na temelju koje je konstruiran sljedeći potisni automat. \cite[str.~85-99]{udzbenik} \cite{auditorne}

\begin{tabular}{|c|c|c|c|c|} \hline
 & a & b & c & $\perp$ \\ \hline
S & \#1 & \#2 & \#3 & \#8 \\ \hline
A & \#2 & \#4 & \#8 & \#4 \\ \hline
B & \#5 & \#4 & \#6 & \#8 \\ \hline
b & \#8 & \#7 & \#8 & \#8 \\ \hline
$\bigtriangledown$ & \#8 & \#8 & \#8 & \#9 \\ \hline
\end{tabular}

\#1: Zamijeni (A); Pomakni; \\
\#2: Zamijeni (S); Pomakni; \\
\#3: Zamijeni (bB); Pomakni; \\
\#4: Izvuci; Zadrži; \\
\#5: Zamijeni (AbB); Pomakni; \\
\#6: Zamijeni (SS); Pomakni; \\
\#7: Izvuci; Pomakni; \\
\#8: Odbaci; \\
\#9: Prihvati; \\
%aud/prim 10

% 2007 - 2MI - 10 - (3)
\item
Nacrtajte DKA na temelju kojeg je izgrađen zadani LR-parser.
Pomoću zadanog LR-parsera parsirajte niz: \texttt{cbacbaba}. \cite[str.~140-144]{udzbenik}

\begin{tabular}{| c || c | c | c | c || c | c | c | } \hline
& \multicolumn{4}{|c||}{Akcija} & \multicolumn{3}{|c|}{Novo stanje} \\ \hline
& a & b & c & $\perp$ & S & A & B \\ \hline
0 & & & p3 & & & s1 & \\ \hline
1 & & p2 & & & & & \\ \hline
2 & & & & r1 & & & \\ \hline
3 & & p4 & & & & & \\ \hline
4 & p7 & & p6 & & & & s5 \\ \hline
5 & & r2 & & & & & \\ \hline
6 & & r3 & & & & & \\ \hline
7 & & & & & s8 & & \\ \hline
8 & & p9 & & & & & \\ \hline
9 & & r4 & & & & & \\ \hline
\end{tabular}

pX	= Pomakni (X); \\
r1	= Reduciraj (S \(\to\) Ab); Prihvati (); \\
r2	= Reduciraj (A \(\to\) cbB); \\
r3	= Reduciraj (B \(\to\) c); \\
r4	= Reduciraj (S \(\to\) aAb); \\
sX	= Stavi (X); \\ 

% AV - 10 - (3)
\item
Prikažite korake tijekom parsiranja niza \texttt{bbbcb} primjenom zadanog LR(1)-parsera. \cite[str.~141-144]{udzbenik} \cite{auditorne}\\

\begin{tabular}{| c || c | c | c || c | c | c | } \hline
Stanje & \multicolumn{3}{|c||}{Akcija} & \multicolumn{3}{|c|}{Novo stanje} \\ \hline
& b & c & $\perp$ & $<$S$>$ & $<$A$>$ & $<$B$>$ \\ \hline
0 & P1 & & & & & \\ \hline
1 & P2 & & & & S3 & \\ \hline
2 & P5 & & & & & S6 \\ \hline
3 & P4 & & & & & S7 \\ \hline
4 & & & R3 & & & \\ \hline
5 & & R3 & & & & \\ \hline
6 & & P8 & & & & \\ \hline
7 & & & Prihvati & & & \\ \hline
8 & R2 & & & & & \\ \hline
\end{tabular}

\begin{alltt}
R1 = Reduciraj ( <S> \(\to\) b <A> <B> )
R2 = Reduciraj ( <A> \(\to\) b <B> c )
R3 = Reduciraj ( <B> \(\to\) b )
\end{alltt}
%aud/prim 12

% 2003 - 2MI - 7 - (3)
\item
Izgradite SLR(1)-parser za zadanu gramatiku. \cite[str.~139-147]{udzbenik} \cite{auditorne}

\begin{alltt}
<S> \(\to\) <S><S>* | <S><S>+ | c | v
\end{alltt} 
%aud/prim 13

% 2007 - 3MI - 4 -(3)
\item
Za zadanu gramatiku konstruirajte SLR(1)-parser. \cite[str.~139-147]{udzbenik} \cite{auditorne}

\begin{alltt}
<S> \(\to\) m | q<P>
<P> \(\to\) <Q><P>
<Q> \(\to\) b
\end{alltt} 
%aud/prim 13

% 2002 - 2MI - 4 - (3)
\item
Konstruirajte SLR(1)-parser za zadanu gramatiku. \cite[str.~139-147]{udzbenik} \cite{auditorne}

\begin{alltt}
<S> \(\to\) <A>c<C>e
<A> \(\to\) a<B> | b<B>
<B> \(\to\) <A> | \(\varepsilon\)
<C> \(\to\) df<C> | \(\varepsilon\)
\end{alltt} 
%aud/prim 13

% 2000 - 2MI - 10 - (3)
\item
Za zadanu gramatiku konstruirajte SLR(1)-parser. \cite[str.~139-147]{udzbenik} \cite{auditorne}

\noindent
\begin{alltt}
<S> \(\to\) <A><B> | a<S>
<A> \(\to\) a<A>a | ac
<B> \(\to\) b<A>
\end{alltt} 
%aud/prim 13

% 2001 - 2MI - 6 - (3)
\item
Konstruirajte SLR(1)-parser za zadanu gramatiku. \cite[str.~139-147]{udzbenik} \cite{auditorne}

\noindent
\begin{alltt} 
<S> \(\to\) <B>a | a<A><S>
<A> \(\to\) <A>c | cb
<B> \(\to\) a<S>b | b<A>
\end{alltt} 
%aud/prim 13

% AV - 11 - (3)
\item
Za zadanu gramatiku izgradite SLR(1)-parser. \cite[str.~139-147]{udzbenik} \cite{auditorne}

\begin{alltt}
<S> \(\to\) a<A>c
<A> \(\to\) x<S> | \(\varepsilon\)
\end{alltt}
%aud/prim 13

% 1999 - 2MI - 9 - (3)
\item 
Konstruirajte SLR(1)-parser za zadanu gramatiku. \cite[str.~139-147]{udzbenik} \cite{auditorne}

\begin{alltt}
<S> \(\to\) <B> | a<B>b<A>
<A> \(\to\) <B>b<A> | c
<B> \(\to\) d | e<S><C>
<C> \(\to\) f<S> | \(\varepsilon\)
\end{alltt}
%aud/prim 13

% 2009 - P2MI-E - 5 - (3)
% 2009 - P2MI - 10 - (3)
% 2010 - 2MI - 10 - (3)
\item
Izgradite SLR(1)-parser za zadanu gramatiku. \cite[str.~139-147]{udzbenik} \cite{auditorne}

\begin{alltt}
<S> \(\to\) <A>a<B>b
<A> \(\to\) a
<B> \(\to\) c<A> | \(\varepsilon\)
\end{alltt}
%aud/prim 13

% 2004 - 2MI - 7 - (3)
\item
Za zadanu gramatiku izgradite SLR(1)-parser. \cite[str.~139-147]{udzbenik} \cite{auditorne}

\begin{alltt}
<S> \(\to\) <B><A>
<A> \(\to\) <B>b<A> | a<B> | b
<B> \(\to\) c<A>
\end{alltt} 
%aud/prim 13

% 2002 - 3MI - 3 - (3)
\item
Izgradite LR(0)-parser za zadanu gramatiku. \cite[str.~139-147]{udzbenik} \cite{auditorne}

\begin{alltt}
<S> \(\to\) <S>a<S> | b<S>c | d
\end{alltt}
%aud/prim 13

% 2003 - 3MI - 7 - (3)
\item
Konstruirajte LR(1)-parser za zadanu gramatiku. \cite[str.~147-152]{udzbenik} \cite{auditorne}

\begin{alltt}
<S> \(\to\) a<A>b<A> | ba<A>
<A> \(\to\) a<A> | \(\varepsilon\)
\end{alltt}
%aud/prim 14

% 2009 - 2MI - 10 - (3)
\item
Izgradite LR(1)-parser za zadanu gramatiku. \cite[str.~139-147]{udzbenik} \cite{auditorne}

\begin{alltt}
<S> \(\to\) b<A>
<A> \(\to\) <S>a | \(\varepsilon\)
\end{alltt}
%aud/prim 13

% 2009 - P2MI-L - 10 - (3)
\item
Za zadanu gramatiku izgradite LR(1)-parser. \cite[str.~139-147]{udzbenik} \cite{auditorne}

\begin{alltt}
<S> \(\to\) <S>a | a<A>b
<A> \(\to\) ab<A> | b<B><B>
<B> \(\to\) <A><B> | ab
\end{alltt}
%aud/prim 13

% AV - 12 - (3)
\item
Za zadanu gramatiku izgradite LR(1)-parser. \cite[str.~139-147]{udzbenik} \cite{auditorne}

\begin{alltt}
<S> \(\to\) b<A><B>
<A> \(\to\) b<B>c
<B> \(\to\) b
\end{alltt}
%aud/prim 13

% 2001 - 3MI - 1 - (3)
\item
Konstruirajte kanonski LR-parser (LR(1)) za zadanu gramatiku. \cite[str.~139-147]{udzbenik} \cite{auditorne}

\begin{alltt}
<S> \(\to\) a<A>c | b<A>d
<A> \(\to\) a<A> | b<A> | \(\varepsilon\)
\end{alltt} 
%aud/prim 13

% 2010 - P2MI - 10 - (3)
% 2010 - P3MI - 9 - (3)
\item
Izgradite LALR(1)-parser za zadanu gramatiku.  \cite[str.~155-156]{udzbenik}

\begin{alltt}
<S> \(\to\) <A>a<B>b
<A> \(\to\) a
<B> \(\to\) c<A> | \(\varepsilon\)
\end{alltt}

\end{enumerate}

%%%%%%%%%%%%%%%%%%%%%%%%%%%%%%%%%%%%%%%%%%%%%%%%%%%%%%%%%%%%%%%%%%%%%%%%%%%%%%%

\chapter{Semantička analiza}

%%%%%%%%%%%%%%%%%%%%%%%%%%%%%%%%%%%%%%%%%%%%%%%%%%%%%%%%%%%%%%%%%%%%%%%%%%%%%%%

\begin{enumerate}[resume]

% 1999 - 3MI - 7 - (4)
\item 
Proširite sljedeću gramatiku svojstvima i akcijskim znakovima tako da se dobije L-atributna prijevodna gramatika koja će računati dekadsku vrijednost rimskih brojeva.
Početni znak gramatike \texttt{<S>} neka ima samo jedno i to izvedeno svojstvo.
Konstruirajte potisni automat za dobivenu gramatiku. \cite[str.~180-195]{udzbenik} \cite{auditorne}

\noindent
\begin{alltt}
<S> \(\to\) V<A> | I<B>
<A> \(\to\) I<C> | \(\varepsilon\)
<B> \(\to\) V | I<D> | \(\varepsilon\)
<C> \(\to\) I<D> | \(\varepsilon\)
<D> \(\to\) I | \(\varepsilon\)
\end{alltt}
%aud/prim 15,16

% 2001 - 3MI - 5 - (4)
\item
Zadana je gramatika koja generira aritmetičke izraze čiji su operandi razlomci zapisani u obliku \texttt{brojnik:nazivnik}.
Pretvorite zadanu gramatiku u LL(1)-gramatiku.
Dobivenu gramatiku proširite svojstvima i akcijskim znakovima tako da se dobije L-atributna gramatika koja izračunava i potom ispisuje vrijednost aritmetičkog izraza.
Definirajte akcijske znakove pomoću: cjelobrojnog dijeljenja, množenja i zbrajanja, funkcije \texttt{zajnaz} koja izračunava zajednički nazivnik i funkcije \texttt{ispis} koja ispisuje razlomak.

\begin{alltt}
<S> \(\to\) <E>
<E> \(\to\) <E>+<T> | <T>
<T> \(\to\) <P> | <T>*<P>
<P> \(\to\) (<E>) | broj:broj
\end{alltt}

Znak \texttt{:} nije operator dijeljenja, već specijalni znak koji odvaja brojnik od nazivnika.
U produkcijama je \texttt{broj} završni znak koji predstavlja cijeli broj.
Niti u jednom trenutku razlomak se ne smije pretvoriti u realni broj, već se tijekom izračunavanja vrijednosti izraza razlomci promatraju kao par cjelobrojnih vrijednosti (brojnik i nazivnik). \cite[str.~107-111, 180-183]{udzbenik} \cite{auditorne}
%aud/prim 15

% 2002 - 2MI - 6 - (4)
\item
Zadana je L-atributna gramatika koja izračunava vrijednost jednostavnih izraza koji se sastoje od operacija zbrajanja i množenja.
Završni znak broj cjelobrojna je konstanta čije svojstvo odgovara vrijednosti dekadskog broja. \cite[str.~180-198]{udzbenik} \cite{auditorne}

a) odredite nasljedna i izvedena svojstva te odrediti redoslijed računanja svojstava i domene za računanje svojstava svake produkcije,

b) definirajte akcije potisnog automata i pokazati promjene na stogu potisnog automata za ulazni niz \texttt{6+2*4},

c) napišite parser metodom rekurzivnog spusta; trenutni znak na ulazu je pohranjen u globalnu varijablu znak, a sljedeći znak se u tu varijablu čita pozivom globalne funkcije \texttt{noviznak()}; pretpostaviti da za izvođenje akcijskih znakova već postoje odgovarajući potprogrami: \texttt{Zbroji()} i \texttt{Pomnozi()}.

\begin{tabular}{ l l }

\(S_{a} \to U_{b} T_{c,d}\) & \(c \leftarrow b, a \leftarrow d \) \\
\(T_{a,b} \to + U_{c} \{Zbroji\}_{d,e,f} T_{g,h} \) & \( d \leftarrow a, e \leftarrow c, g \leftarrow f, b \leftarrow h \) \\
\(T_{a,b} \to \varepsilon \) & \( b \leftarrow a \) \\
\(U_{a} \to P_{b} V_{c,d} \) & \( c \leftarrow b, a \leftarrow d \) \\
\(V_{a,b} \to * P_{c} \{Pomnozi\}_{d,e,f} V_{g,h} \) & \( d \leftarrow a, e \leftarrow c, g \leftarrow f, b \leftarrow h \) \\
\(V_{a,b} \to \varepsilon \) & \( b \leftarrow a \) \\
\(P_{a} \to ( S_{b} ) \) & \( a \leftarrow b \) \\
\(P_{a} \to broj_{b} \) & \( a \leftarrow b \) \\

\end{tabular}
%aud/prim 15,16

% 2002 - 3MI - 7 - (4)
\item
Zadanu gramatiku koja prihvaća oktalne brojeve s pomičnim zarezom proširite svojstvima i akcijskim znakovima koji će izračunati dekadsku vrijednost pročitanog broja (također s pomičnim zarezom).
Pretpostavite da završni znak \texttt{znamenka} ima jedno svojstvo koje je izvedeno i predstavlja vrijednost oktalne znamenke (\texttt{0}--\texttt{7}). \cite[str.~177-180]{udzbenik} \cite{auditorne}

\begin{alltt}
<S> \(\to\) znamenka <A> . <B>
<A> \(\to\) znamenka <A> | \(\varepsilon\)
<B> \(\to\) znamenka <B> | \(\varepsilon\)
\end{alltt} 
%aud/prim 15

% 2003 - 3MI - 1 - (4)
\item
Odredite koja su izvedena, a koja nasljedna svojstva te redoslijed računanja svojstava za sve produkcije zadane L-atributne gramatike. \cite[str.~173-181]{udzbenik}

\begin{tabular}{ ll } 

\( S_{a,b} \to x_{c} A_{d,e} \{Akc1\}_{f,g} B_{h,i,j} z_{k} \{Akc2\}_{m,n,o} \) & d \(\leftarrow\) a, e \(\leftarrow\) c, f \(\leftarrow\) c, h \(\leftarrow\) g, \\ & m \(\leftarrow\) i, b \(\leftarrow\) i, n \(\leftarrow\) j, o \(\leftarrow\) k \\
\( A_{a,b} \to y_{c} z_{d} A_{e,f} \{Akc3\}_{g,h} \) & e \(\leftarrow\) a, f \(\leftarrow\) c, g \(\leftarrow\) b, h \(\leftarrow\) d \\
\( A_{a,b} \to A_{c,d} z_{e} \{Akc3\}_{f,g} \) & c \(\leftarrow\) a, d \(\leftarrow\) b, f \(\leftarrow\) d, g \(\leftarrow\) e \\
\( A_{a,b} \to x_{c} B_{d,e,f} x_{g} \{Akc4\}_{h,i,j,k,m} \) & d \(\leftarrow\) a, h \(\leftarrow\) b, i \(\leftarrow\) c, j \(\leftarrow\) e, k \(\leftarrow\) f, m \(\leftarrow\) g \\
\( B_{a,b,c} \to z_{d} x_{e} B_{f,g,h} \{Akc5\}_{i,j,k,m} \) & f \(\leftarrow\) e, i \(\leftarrow\) a, j \(\leftarrow\) d, k \(\leftarrow\) g, b \(\leftarrow\) h, c \(\leftarrow\) m \\
\( B_{a,b,c} \to y_{d} \{Akc6\}_{e,f,g} y_{h} \{Akc6\}_{i,j,k} y_{m} \) & e \(\leftarrow\) a, f \(\leftarrow\) d, i \(\leftarrow\) g, j \(\leftarrow\) h, b \(\leftarrow\) m, c \(\leftarrow\) k \\
\( B_{a,b,c} \to S_{d,e} x_{f} \{Akc6\}_{g,h,i} \) & d \(\leftarrow\) a, g \(\leftarrow\) e, h \(\leftarrow\) f, b \(\leftarrow\) e, c \(\leftarrow\) i \\
\end{tabular} 

% 2004 - 2MI - 9 - (4)
\item
Za zadanu L-atributnu gramatiku napišite parser zasnovan na rekurzivnom spustu.
Koristite pseudokôd sličan programskom jeziku C.
U kôdu koristite reference umjesto pokazivača te zanemariti deklaracije varijabli.
Pretpostavite da su izlazne akcije već ostvarene kao zasebni potprogrami.
Na raspolaganju su još i globalna varijabla \texttt{proc\_znak} u kojoj je pohranjen zadnji pročitani ulazni znak, potprogram \texttt{slij\_znak()} koji učitava sljedeći ulazni znak te potprogram \texttt{greska()} koji zaustavlja parsiranje i ispisuje poruku o grešci. \cite[str.~195-198]{udzbenik}

\begin{tabular}{ll} 
\( S_{s} \to \{Zbroji\}_{a,b,c} [ L_{d,e} ] \{Ispisi\}_{f,g} \)  &  s\(\leftarrow\)1 a\(\leftarrow\)s b\(\leftarrow\)1 e\(\leftarrow\)c f\(\leftarrow\)d g\(\leftarrow\)c \\
\( L_{a,b} \to \{Oduzmi\}_{c,d,e} [ L_{f,g} ] \{Ispisi\}_{h,i} \)  &  a\(\leftarrow\)0 c\(\leftarrow\)b d\(\leftarrow\)-1 g\(\leftarrow\)e h\(\leftarrow\)f i\(\leftarrow\)e \\
\(L_{a,b} \to X_{c} L_{d,e} \{Oduzmi\}_{f,g,h} \)  &  a\(\leftarrow\)h e\(\leftarrow\)b f\(\leftarrow\)d g\(\leftarrow\)c \\
\(L_{a,b} \to \varepsilon \)  &  a\(\leftarrow\)0 \\
\(X_{a} \to a \) &  a\(\leftarrow\)-1 \\
\end{tabular} 

% 2009 - P2MI-E - 4 - (4)
% 2009 - P3MI - 6 - (4)
% 2009 - P2MI-L - 9 - (4)
% 2010 - 2MI - 9 - (4)
% 2009 - P2MI - 9 - (4)
\item
U pseudokodu sličnom jeziku C napišite parser zasnovan na metodi rekurzivnog spusta za zadanu gramatiku. \cite[str.~195-198]{udzbenik}

\begin{tabular}{ll} 
\( <S> \to a_{p} <B>_{q, r} <A>_{x, y} b_{z} \{Ispisi_{w}\} \) & \( q \leftarrow p + 3, x \leftarrow r, w \leftarrow z + y \times p \) \\
\( <A>_{p,q} \to \varepsilon	 \) & \( q \leftarrow p \% 3 \) \\
\( <B>_{p, q} \to a_{r} <A>_{x, y} b_{z} \)& \(x \leftarrow p, q \leftarrow y \times z + r \) \\
\end{tabular} 

% 2007 - 2MI - 2 - (3)
\item
Navedite sve ulazne nizove za koje zadana gramatika generira sljedeći niz izlaznih završnih znakova: \verb|{|x\verb|}|\verb|{|q\verb|}|\verb|{|b\verb|}|\verb|{|q\verb|}|\verb|{|w\verb|}|\verb|{|a\verb|}|. Za svaki ulazni niz nacrtajte sintaksno stablo. \cite[str.~171]{udzbenik}

\begin{alltt}
<S>	\(\to\)	b \verb|{|z\verb|}| a<A> | <A> \verb|{|w\verb|}| b \verb|{|a\verb|}| a
<A>	\(\to\)	\verb|{|x\verb|}| <B>c<B><A> \verb|{|q\verb|}| | c \verb|{|y\verb|}| \verb|{|z\verb|}| a <A> b \verb|{|p\verb|}| | a
<B>	\(\to\)	<B> a \verb|{|q\verb|}| c b \verb|{|b\verb|}| <A> | b a \verb|{|q\verb|}| <A> c \verb|{|b\verb|}| | c
\end{alltt} 

% 2004 - 3MI - 3 - (4)
\item
Zadana je gramatika koja opisuje deklaraciju i inicijalizaciju dvodimenzionalnog polja.
Odredite PRIMIJENI skupove za sve produkcije.
Proširite gramatiku svojstvima i akcijskim znakovima tako da se omogući provjera ispravnosti deklaracije i inicijalizacije.
Osim operatora \texttt{=},\texttt{[},\texttt{]},\verb|{|,\verb|}|,\texttt{;},\texttt{s\_zarez}, u završne znakove spadaju i ključne riječi \texttt{kr\_int}, \texttt{kr\_char}, \texttt{kr\_double}, identifikatori \texttt{idn} te konstante \texttt{const}.
Pretpostavite da završni znak \texttt{const} ima jedno izvedeno svojstvo koje označava jedan od tri moguća tipa konstante.
Veličina polja može se zadati samo pomoću cjelobrojnih konstanti.
Tijekom provjere nije potrebno ispitivati ispravnost dimenzija inicijalizacijskog dijela. \cite[str.~178-180]{udzbenik}

\begin{alltt}
<S> \(\to\) <T> idn <D> = <I> ;
<D> \(\to\) [ const ][ const ]
<E> \(\to\) \verb|{| <T> <K> \verb|}|
<I> \(\to\) \verb|{| <E> <L> \verb|}|
<K> \(\to\) s_zarez <T> <K> | \(\varepsilon\)
<L> \(\to\) s_zarez <E> <L> | \(\varepsilon\)
<T> \(\to\) const | kr_char | kr_double | kr_int
\end{alltt} 

% 2007 - 2MI - 7 - (4)
\item
Gramatiku koja služi za parsiranje naredbe deklaracije s pridruživanjem proširite svojstvima i akcijskim znakovima tako da se provjerava ispravnost tipa s lijeve i desne strane operatora pridruživanja.
Nadalje, izračunajte i ispišite vrijednost konstante koja se zadanim izrazom pridružuje deklariranoj varijabli.
Jezik ne dopušta implicitnu pretvorbu tipova. \cite[str.~178-180]{udzbenik} \cite{auditorne}

\begin{alltt}
<S>	\(\to\) <T> <I> = <E> ;
<T>	\(\to\) int | float | double
<I>	\(\to\) idn
<V>	\(\to\) c\_int | c\_float | c\_double
<E>	\(\to\) <V> <L> | <C> ( <E> )
<L>	\(\to\) <O> <E> <L> | \(\varepsilon\)
<C>	\(\to\) <T> | \(\varepsilon\)
<O>	\(\to\) + | -
\end{alltt}

Napomena: Produkcija \texttt{<E> \(\to\) <C> ( <E> )} omogućava eksplicitnu promjenu tipa izraza \texttt{<E>}. 
%aud/prim 15

% 2007 - 2MI - 8 - (4)
\item
Za zadanu atributnu gramatiku, u pseudokodu sličnom jeziku C napišite parser metodom rekurzivnog spusta. \cite[str.~195-198]{udzbenik}

\begin{tabular} { l l }
\( <S>_{o} \to a <A>_{p} b c <B>_{q, r} \{Ispisi\}_{w} \) & p\(\leftarrow\)o  w\(\leftarrow\)q  r\(\leftarrow\)o \\
\( <S>_{o} \to b <A>_{p} \{Zbroji\}_{r,w,z} \)  &  p\(\leftarrow\)o  w\(\leftarrow\)o  r\(\leftarrow\)o  \\
\( <A>_{o} \to c <B>_{p,q} \{Oduzmi\}_{r,w,z} \)  & r\(\leftarrow\)o  w\(\leftarrow\)p  q\(\leftarrow\)o  \\
\( <B>_{o,p} \to a c \)  &  o\(\leftarrow\)p + 2  \\
\end{tabular} 

% 2009 - 3MI - 10 - (4)
\item
Izgradite atributnu prijevodnu gramatiku koja parsira nizove cijelih dekadskih
brojeva zapisane u obliku \verb|{|\texttt{a1, a2, a3, ..., an}\verb|}|.

Proširite gramatiku svojstvima i akcijskim znakovima koji računaju aritmetičku sredinu zadanog niza tako da sva pravila računanja budu pravila preslikavanja.
Nizovi mogu biti proizvoljne duljine. \cite[str.~178-180]{udzbenik}

% 2010 - P2MI - 9 - (4)
% 2009 - 2MI - 9 - (4)
% 2009 - P2MI - 8 - (4)
% 2009 - P3MI - 7 - (4)
% 2010 - P3MI - 8 - (4)
\item
Izgradite atributnu prijevodnu gramatiku koja parsira nizove koji predstavljaju multiskup (skup u kojem se elementi mogu ponavljati) točaka u ravnini zapisan u obliku

\begin{alltt}
[(x1, y1), (x2, y2), ..., (xn, yn)].
\end{alltt}

U gramatici za broj koji predstavlja \texttt{x} ili \texttt{y} koordinatu točke koristiti završni znak \texttt{b}.
Proširite gramatiku svojstvima i akcijskim znakovima koji računaju koordinate težišta multiskupa tako da sva pravila računanja budu pravila preslikavanja. \cite[str.~178-180]{udzbenik}

% 2004 - 2MI - 6 - (4)
\item
Opišite postupak izgradnje potisnog automata za prijevodnu gramatiku. \cite[str.~184-195]{udzbenik}

% 2004 - 2MI - 8 - (4)
\item
Objasnite što je provjera vrijednosti obilježja i opišite pojedine postupke za provjeru vrijednosti obilježja. \cite[str.~200-202]{udzbenik}

% 2007 - 2MI - 6 - (4)
\item
Opišite svojstva L-atributne prijevodne gramatike. \cite[str.~180-181]{udzbenik}

% 2003 - 2MI - 4 - (4)
\item
Nabrojite i objasnite formalne modele semantičkog analizatora. \cite[str.~169-170]{udzbenik}

% 2003 - 2MI - 8 - (4)
\item
Definirajte L-atributnu prijevodnu gramatiku. \cite[str.~180-181]{udzbenik}

% 2009 - 2MI - 4 - (4)
% 2000 - 2MI - 7 - (4)
\item
Navedite i objasnite tri najčešće primjenjivana formalna modela semantičkog analizatora. \cite[str.~169-170]{udzbenik}

% 2001 - 2MI - 9 - (4)
\item
Objasnite sintaksom vođenu semantičku analizu. \cite[str.~170-171]{udzbenik}

% 2001 - 3MI - 2 - (4)
\item
Definirajte atributnu prijevodnu gramatiku. \cite[str.~173]{udzbenik}

% 2001 - 3MI - 4 - (4)
\item
Opišite algoritam provjere jednakosti tipova obilježja temeljen na provjeri jednakosti strukture obilježja. \cite[str.~204-208]{udzbenik}

% 2002 - 2MI - 5 - (4)
\item
Objasnite kako se obrađuju izvedena svojstva izlaznih završnih znakova koji se ne stavljaju na stog. \cite[str.~173-176]{udzbenik}

% 2002 - 2MI - 7 - (4)
\item
Navedite i objasnite algoritam ispitivanja jednakosti obilježja konstantnih vrijednosti. \cite[str.~203-204]{udzbenik}

% 2009 - 2MI - 5 - (4)
\item
Opišite korake gradnje atributnog generativnog stabla.
Definirajte potpuno atributno generativno stablo. \cite[str.~178]{udzbenik}

% 2009 - P2MI-E - 2 - (4)
% 2003 - 2MI - 6 - (4)
\item
Objasnite kako se obrađuju svojstva izvorišta koja nemaju dostupne vrijednosti. \cite[str.~194]{udzbenik}

% 2010 - 2MI - 5 - (4)
\item
Navedite uvjete pod kojima je atributna prijevodna gramatika ujedno i L-atributna prijevodna gramatika. \cite[str.~180-181]{udzbenik}

% 2010 - P2MI - 2 - (4)
\item
Objasnite razliku između izvedenih i nasljednih svojstava.
Kako se izvedena i nasljedna svojstva spremaju na stog tijekom parsiranja od vrha prema dnu? \cite[str.~173-177]{udzbenik}

% 2010 - P3MI-L - 2 - (4)
\item
Navedite zadatke semantičkog analizatora. \cite[str.~160]{udzbenik}

% 2010 - 3MI - 8 - (4)
\item
Izgradite atributnu prijevodnu gramatiku koja računa zbroj elemenata polja pozitivnih cijelih brojeva.
Polje je zapisano u sljedećem formatu:

\begin{alltt}
[x1 $ x2 $ x3 $ ... $ xn]
\end{alltt}

U gramatici za brojeve \texttt{x} koristiti završni znak \texttt{b}.
Proširite gramatiku svojstvima i akcijskim znakovima koji računaju zbroj elemenata polja tako da sva pravila računanja budu pravila preslikavanja.
Polje može biti proizvoljne veličine.  \cite[str.~177-180]{udzbenik} \cite{auditorne}
%aud/prim 15

% AV - 13 - (4)
\item
Izgradite atributnu prijevodnu gramatiku koja parsira parove binarnih brojeva zapisane u obliku

\begin{alltt}
x1 x2 x3 ... xn \(\diamondsuit\) y1 y2 y3 ... ym
xi, yi \(\in\) \verb|{|0, 1\verb|}|
\end{alltt}

Simbol \(\diamondsuit\) predstavlja operator zbrajanja koji za neparne bitove oba broja uzima vrijednost \texttt{0}.
Bitovi se broje od najmanje značajnoga prema najznačajnijem, počevši od nule.
Na primjer:
\begin{alltt}
01010010 \(\diamondsuit\) 1011011011 = 01010000 + 0001010001
\end{alltt}
Proširite izgrađenu gramatiku svojstvima i akcijskim znakovima koji računaju rezultat primjene operatora \(\diamondsuit\) izražen u dekadskom obliku.
Brojevi mogu imati proizvoljan broj znamenaka. \cite[str.~177-180]{udzbenik} \cite{auditorne}
%aud/prim 15

% AV - 14 - (4)
\item
Izgradite potisni automat za zadanu atributnu prijevodnu gramatiku.
Za sve akcije \emph{Zamijeni} prikazati stanje na stogu neposredno prije i neposredno poslije primjene akcije. \cite[str.~184-195]{udzbenik} \cite{auditorne}

\begin{tabular}{ll}
(1) \( <S> \to a_{p} b_{q} <A>_{r} \{X_{v}\} \) &   \( v \leftarrow p \times q+r \) \\
(2) \( <A>_{p} \to a_{q} <B>_{r} \) & \( p \leftarrow q+r \) \\
(3) \( <B>_{p} \to c_{q} \) & \( p \leftarrow q \)  \\
\end{tabular} 
%aud/prim 16

\end{enumerate}

%%%%%%%%%%%%%%%%%%%%%%%%%%%%%%%%%%%%%%%%%%%%%%%%%%%%%%%%%%%%%%%%%%%%%%%%%%%%%%%

\chapter{Potpora izvođenju ciljnog programa}

%%%%%%%%%%%%%%%%%%%%%%%%%%%%%%%%%%%%%%%%%%%%%%%%%%%%%%%%%%%%%%%%%%%%%%%%%%%%%%%

\begin{enumerate}[resume]

% 2001 - 3MI - 8 - (5)
\item
Za svaki od četiri načina prenošenja parametara odredite stanja varijabli \texttt{j}, \texttt{k} i \texttt{l} te polja \texttt{i} nakon izvođenja zadanog programskog odsječka. \cite[str.~243-252]{udzbenik}

\begin{alltt}
int i[3]=\verb|{|5,6,7\verb|}|,j=8,k=2,l=0;

divmod(int a,int b,int c,int d)
\verb|{|
  c=a/b;  
  d=a\%b;
  l=a\%b+1;
\verb|}|

divmod(i[k],j,k,l);
\end{alltt} 

% 2003 - 3MI - 3 - (5)
% 2009 - P2MI-L - 7 - (5)
\item
Za dani programski odsječak odredite ispis ako se kod poziva potprograma koristi: \cite[str.~243-252]{udzbenik}\\
a) razmjena vrijednosti \\
b) razmjena adresa (kod rekurzivnog poziva šalje se ista adresa koja je primljena kao parametar) \\
c) razmjena imena

\begin{alltt}
f(a,b)
\verb|{|
   ispiši(a,b);
   ako (b>=1)
   \verb|{|
      b=b-1;
      a=a-1;
      f(b,a);
   \verb|}|
\verb|}|

glavni()
\verb|{|
   cijeli x[3]=\verb|{|1,0,1\verb|}|;
   cijeli y=2;

   f(y,x[y]);
   ispiši(x[0],x[1],x[2],y);
\verb|}|
\end{alltt} 

% 2002 - 3MI - 2 - (5)
\item
Opišite algoritam gradnje lanca kazaljki nelokalnih imena i vektora dubine gniježđenja kod statičkog pravila djelokruga ugniježđenih procedura. \cite[str.~239]{udzbenik}

% 2003 - 3MI - 2 - (5)
\item
Objasnite povezivanje imena izvornog programa i objekata ciljnog programa te relaciju okoline i relaciju stanja. \cite[str.~221-222]{udzbenik}

% 2010 - 3MI - 2 - (5)
\item
Ukratko definirajte relaciju okoline i relaciju stanja. \cite[str.~221-222]{udzbenik}

% 2001 - 3MI - 6 - (5)
\item
Objasnite načine ostvarenja dinamičkog pravila djelokruga. \cite[str.~241-242]{udzbenik}

% 2003 - 3MI - 4 - (5)
\item
Opišite mehanizam povratne razmjene vrijednosti parametara procedura te navedite način ostvarenja. \cite[str.~243-252]{udzbenik}

% 2004 - 3MI - 2 - (5)
\item
Navedite i kratko objasnite postupke za određivanje djelokruga deklaracije nelokalnih imena. \cite[str.~236-241]{udzbenik}
% 2007 - 3MI - 9 - (5)

\item
Objasnite način ostvarenja statičkog pravila djelokruga nelokalnih imena ugniježđenih procedura. \cite[str.~236-241]{udzbenik}

% 2009 - 3MI - 2 - (5)
% 2010 - P3MI-L - 5 - (5)
\item
Navedite osnovne načine razmjene ulazno/izlaznih parametara procedura i što se zapisuje u opisnik pozvane procedure prilikom pojedinog načina razmjene. \cite[str.~243-244]{udzbenik}

% 2009 - 3MI - 3 - (5)
\item
Objasnite pojmove djelokrug deklaracije i životni vijek pridruživanja imena.
Što se događa sa životnim vijekom pridruživanja imena prilikom rekurzivnih poziva potprograma? \cite[str.~233-234]{udzbenik}

% 2010 - 3MI - 3 - (5)
\item
Objasnite djelokrug deklaracije i navedite moguća pravila definiranja djelokruga deklaracija.
Pravila nije potrebno objašnjavati. \cite[str.~233-234]{udzbenik}

% 2010 - P3MI - 3 - (5)
\item
Objasnite vektor dubine gniježđenja i algoritam njegove izgradnje. \cite[str.~239]{udzbenik}

% 2007 - 3MI - 10 - (5)
\item
Za prikazani programski odsječak odredite ispis ako se kod poziva potprograma koristi: (a) razmjena vrijednosti, (b) razmjena adresa, (c) razmjena imena i (d) povratna razmjena vrijednosti. \cite[str.~243-252]{udzbenik} \cite{auditorne}

\begin{alltt}
varijabla x=0, y=3, z=-1;
polje o[3]=10, o[4]=20;
  Racunaj(p, q, r) \verb|{|
   z = p + x
   q = q + 1
   Ispisi(p, q, r);
   r = z + q
  \verb|}|
\verb|{|
  za x = 3 do 4 \verb|{|
    Racunaj(o[x], o[3+x\%2], z);
    Ispisi(x, y, z, o[3], o[4]);
  \verb|}|
\verb|}|
\end{alltt} 
%aud/prim 19

% 2009 - 3MI - 9 - (5)
\item
Za zadani program prikažite sadržaj opisinka procedura u trenutku neposredno prije izvođenja naredbe \texttt{05} ako se koristi: (a) statičko pravilo djelokruga, (b) dinamičko pravilo djelokruga.
U oba slučaja objasnite tijek izvođenja programa i prikažite što će se ispisati kao posljedica naredbe \texttt{17}. \cite[str.~234-242]{udzbenik} \cite{auditorne}

\begin{alltt}
01  Glavni()
02    int y = 3;
03    def Z(a)
04      def X(x)\verb|{|
05        vrati x+y;
06      \verb|}|
07      def Y(y)\verb|{|
08        ako y <= 4 onda
09          vrati Y(5);
10        inače
11          vrati X(y);
12      \verb|}|
13    \verb|{|
14      Y(a);
15    \verb|}|
16  \verb|{|
17    ispiši Z(y)
18  \verb|}|
\end{alltt} 
%aud/prim 18

% AV - 16 - (5)
\item
Za zadani program prikažite sadržaj opisnika procedura u trenutku prije izvođenja naredbe \texttt{07} ako se koristi: (a) statičko pravilo djelokruga, (b) dinamičko pravilo djelokruga. \cite[str.~234-242]{udzbenik} \cite{auditorne}

\begin{alltt}
01  Glavni()
02    int y = 3;
03    def Z(a)
04      int r = 5
05      def X(x)
06      \verb|{|
07        vrati x+y+1;
08      \verb|}|
09      def Y(y)
10      \verb|{|
11        vrati X(y)+1;
12      \verb|}|
13    \verb|{|
14      Y(a);
15    \verb|}|
16  \verb|{|
17    ispiši Z(y+1)
18  \verb|}|
\end{alltt} 
%aud/prim 18

% 2010 - 3MI - 6 - (5)
\item
Prikažite i objasnite izvođenje sljedećeg programa ako se za poziv procedure koristi (a) razmjena vrijednosti, (b) razmjena adresa i (c) razmjena imena \cite[str.~243-252]{udzbenik} \cite{auditorne}

\begin{alltt}
01  varijabla a = 0;
02  polje V = \verb|{|7, 8\verb|}|; // V[0]=7, V[1]=8
03  procedura Proc(x, y) \verb|{|
04    x = y;
05    a = 1;
06    y = a;
07  \verb|}|
08  \verb|{| // ovo je glavni program
09    Proc(V[0], V[a]);
10    Ispiši(a, V[0], V[1]);
11  \verb|}|
\end{alltt} 
%aud/prim 19

% AV - 15 - (5)
\item
Za zadani program izgradite stablo aktiviranja procedura. \cite[str.~228-229]{udzbenik} \cite{auditorne}

\begin{alltt}
01  Glavni()
02    X(a)
03    \verb|{|
04      vrati a + 1;
05    \verb|}|
06    Y(b, c)
07    \verb|{|
08      vrati c - b/4;
09    \verb|}|
10
11    Z(d, e)
12    \verb|{|
13      dok (d <= e)
14      \verb|{|
15        d = X(d);
16        e = Y(d, e);
17        Z(d, e);
18        if (d == 5)
19        \verb|{|
20          d = 8;
21          e = 8;
22          dalje;
23        \verb|}|
24      \verb|}|
25    \verb|}|
26  \verb|{|
27    Z(3, 7)
28  \verb|}|
\end{alltt}
%aud/prim 17

% AV - 17 - (5)
\item
Prikažite razliku između razmjene parametara primjenom mehanizma razmjene adresa i mehanizma razmjene imena na sljedećem programu: \cite[str.~243-252]{udzbenik} \cite{auditorne}

\begin{alltt}
01  var x = 0
02  polje A = \verb|{|10, 20\verb|}| // A[0]=10, A[1]=20
03  P(a) \verb|{|
04    x = 1
05    a = 100
06    Ispisi(A[0], A[1])
07  \verb|}|
08  \verb|{|
09    P(A[x])
10  \verb|}|
\end{alltt}
%aud/prim 19

% AV - 18 - (5)
\item
Za zadani program prikažite vrijednosti globalnih i lokalnih varijabli tijekom izvođenja programa. Razmjena parametara procedura ostvaruje se primjenom mehanizma razmjene imena. \cite[str.~243-252]{udzbenik} \cite{auditorne}

\begin{alltt}
01  varijabla x=0, y=3, z=-1;
02  polje o = \verb|{|0, 0, 0, 10, 20\verb|}|;
03  Racunaj(p, q, r) \verb|{|
04    z = p + x;
05    z = (q + 1) \% 2 + 3;
06    Ispisi(p, x, r);
07    r = z + q;
08  \verb|}|
09  \verb|{|
10    za x = 3 do 4 \verb|{|
11      Racunaj(o[x], o[3+x\%2], z);
12      Ispisi(x, y, z, o[3], o[4]);
13    \verb|}|
14  \verb|}|
\end{alltt}
%aud/prim 19

\end{enumerate}

%%%%%%%%%%%%%%%%%%%%%%%%%%%%%%%%%%%%%%%%%%%%%%%%%%%%%%%%%%%%%%%%%%%%%%%%%%%%%%%

\chapter{Generiranje međukoda}

%%%%%%%%%%%%%%%%%%%%%%%%%%%%%%%%%%%%%%%%%%%%%%%%%%%%%%%%%%%%%%%%%%%%%%%%%%%%%%%

\begin{enumerate}[resume]

% 2002 - 3MI - 4 - (6)
\item
Navedite i kratko opišite linearne oblike međukôda. \cite[str.~257-259]{udzbenik}

% 2004 - 3MI - 4 - (6)

\item
Navedite osnovne razine međukoda i objasnite namjenu svake razine. \cite[str.~254-255]{udzbenik}

% 2009 - 3MI - 4 - (6)
\item
Objasnite graf zavisnosti. \cite[str.~260-261]{udzbenik}

% 2004 - 3MI - 7 - (6)
\item
Za zadani programski odsječak nacrtajte grafove zavisnosti. Grafove zavisnosti prikažite u ovisnosti o parametru X za sljedeće slučajeve: \cite[str.~257-259]{udzbenik} \cite{auditorne}\\
a)	X=a (jezični pretprocesor zamjenjuje simboličko ime X varijablom a) \\
b)	X=b (jezični pretprocesor zamjenjuje simboličko ime X varijablom b)

\begin{alltt}
a = b + 5;
dok ( X == 10 )
  b = X + 3;
X = 20;
\end{alltt} 
%aud/prim 22

% 2007 - 3MI - 2 - (6)
% 2009 - P2MI-L - 3 - (6)
\item
Navedite oblike međukôda te za svaki oblik navedite primjere. \cite[str.~255-261]{udzbenik}

% 2009 - 3MI - 6 - (6)
\item
Za zadani program izgradite graf tijeka izvođenja. \cite[str.~257-259]{udzbenik} \cite{auditorne}

\begin{alltt}
01      Učitaj(x)
02      p := 1.0
03  L1: t := abs(p*p-x)
04      if t <= 1e-9 goto L2
05      p := avg(p, x/p)
06      goto L1
07  L2: Ispiši(p)
\end{alltt} 
%aud/prim 22

% 2009 - P2MI-L - 8 - (6)
\item
Za zadani programski odsječak izgradite graf zavisnosti programa koji pokazuje četiri vrste zavisnosti podataka i zavisnosti upravljačkog tijeka izvođenja programa. \cite[str.~257-259]{udzbenik} \cite{auditorne}

\begin{alltt}
a=3+b; c=d+a;
ako (b<c) \verb|{|
  b=b+a; d=3/(b+2);
\verb|}| 
inače ako (b=c)\verb|{|
  b=b+a; d=3/(b+2);
\verb|}|
c = a + d;
\end{alltt}
%aud/prim 22

% 2010 - 3MI - 9 - (6)
\item
Za prikazani isječak programa nacrtajte graf zavisnosti upravljačkog tijeka i graf zavisnosti podataka. \cite[str.~257-259]{udzbenik} \cite{auditorne}

\begin{alltt}
01  p = 1
02  i = 20;
03  a = i / 4
04  ako(i >= 3)
05  \verb|{|
06    q = q + i/a;
07    ako(a<10)
08    \verb|{|
09      a = q*p;
10    \verb|}|
11    i = i-1;
12  \verb|}|
13  p = 3 * a;
\end{alltt} 
%aud/prim 22

% 2001 - 3MI - 7 - (6)
\item
Opišite graf zavisnosti programa i navedite sve zavisnosti koje se njime opisuju. \cite[str.~257-259]{udzbenik}

% 2002 - 3MI - 9 - (6)
\item
Za zadani isječak programa nacrtajte graf zavisnosti upravljačkog tijeka i graf zavisnosti podataka. \cite[str.~257-259]{udzbenik} \cite{auditorne}

\begin{alltt}
a=b+c;
c=a;
ako (b>c)
\verb|{|
  c=b;
  a=b+a;
\verb|}|
inače
\verb|{|
  n=m+a;
  m=n+a;
\verb|}|
n=b+a;
\end{alltt} 
%aud/prim 22

% 2007 - 3MI - 7 - (6)
\item
Za prikazani isječak programa nacrtajte graf zavisnosti upravljačkog tijeka i graf zavisnosti podataka. \cite[str.~257-259]{udzbenik} \cite{auditorne}

\begin{alltt}
p = 1
i = 20;
a = i / 4
dok(i >= 3)
\verb|{|
  q = q + i/a;
  if(a<10)
  \verb|{|
    a = q*p;
  \verb|}|
  i = i-1;
\verb|}|
p = 3 * a;
\end{alltt} 
%aud/prim 22

% 1999 - 3MI - 2 - (6)
\item 
Za dani program nacrtajte graf zavisnosti kako za upravljački tijek, tako i za sve vrste zavisnosti podataka.\cite[str.~257-259]{udzbenik} \cite{auditorne}

\noindent
\begin{alltt}
b:=b*3;
a:=b+d;
ako a<143 tada
  c:=a%8;
  a:=a/8;
d:=b+d;
\end{alltt} 
%aud/prim 22

% 2010 - P3MI-L - 3 - (6)
\item
Navedite tri oblika međukoda. \cite[str.~255]{udzbenik}

% AV - 20 - (6)
\item
Za zadani program izgradite graf tijeka izvođenja. \cite[str.~257-259]{udzbenik} \cite{auditorne}

\begin{alltt}
01      Input(n)
02      Input(p)
03      a0 := 2
04      if n <= 5 goto L1
05      if p > 5 goto L2
06  L1: a1 := a0 + 3
07      a2 := a1 + p
08      a3 := a1 * n
09      goto Z
10  L2: a2 := 3 * 3
11      Output (a3)
12      p := p + 1
13      goto L1
14  Z:  nop
\end{alltt} 
%aud/prim 22

% 2010 - P3MI-L - 9 - (6)
\item
Izgradite atributnu prijevodnu gramatiku koja generira troadresne naredbe za računanje aritmetičkih izraza koji sadrže operatore \texttt{+} i \texttt{*}.
U izrazima je dopušteno korištenje zagrada. \cite[str.~178-180]{udzbenik} \cite{auditorne}
%aud/prim 15

% 2007 - 3MI - 6 - (6)
\item
Na temelju zadane gramatike izgradite atributnu prijevodnu gramatiku za generiranje troadresnih naredbi koje ostvaruju programe napisane u jeziku zadane gramatike. \cite[str.~178-180]{udzbenik}

\begin{alltt}
<S> \(\to\) <N> <S> | \(\varepsilon\)
<N> \(\to\) <Assign> | <If> | <While>
<If> \(\to\) if <Exp> then <N> else <N> end
<While> \(\to\) while <Exp> do <N> end
<Assign> \(\to\) idn '=' <Exp> ';'
<Exp> \(\to\) <Exp> \( \wedge \) <Exp>
<Exp> \(\to\) <Exp> \( \vee \) <Exp>
<Exp> \(\to\) idn
\end{alltt} 

% AV - 19 - (6)
\item
Izgradite atributnu prijevodnu gramatiku koja generira troadresne naredbe za računanje logičkih izraza koji sadrže operator \(\wedge\), \(\vee\) i \(\lnot\). \cite[str.~178-180]{udzbenik}

\end{enumerate}

%%%%%%%%%%%%%%%%%%%%%%%%%%%%%%%%%%%%%%%%%%%%%%%%%%%%%%%%%%%%%%%%%%%%%%%%%%%%%%%

\chapter{Generiranje ciljnog programa}

%%%%%%%%%%%%%%%%%%%%%%%%%%%%%%%%%%%%%%%%%%%%%%%%%%%%%%%%%%%%%%%%%%%%%%%%%%%%%%%

\begin{enumerate}[resume]

% 1999 - 3MI - 5 - (7)
\item 
Napišite tablice generatora ciljnog programa koji kao ulaz koristi sintaksno stablo čije čvorove čine četiri matematičke operacije (zbrajanje, oduzimanje, množenje i dijeljenje).
Ciljno računalo posjeduje samo jedan registar. 
Nacrtajte sintaksno stablo za izraz \texttt{(a/b+(c+d)*(e-f))*(g-h/i)} i navedite redoslijed kojim generator obilazi čvorove. \cite[str.~281-284]{udzbenik}

% 2002 - 3MI - 8 - (7)
\item
Objasnite generiranje ciljnog programa na temelju postfiksnog sustava oznaka. \cite[str.~279-280]{udzbenik}

% 2003 - 3MI - 8 - (7)
\item
Opišite postupak izrade adresa naredbama. \cite[str.~268]{udzbenik}

% 2010 - P3MI - 4 - (7)
\item
Opišite Chaitinov heuristički postupak za bojanje grafa zavisnosti simboličkih i stvarnih registara. \cite[str.~273]{udzbenik}

% 2010 - P3MI - 5 - (7)
\item
Opišite algoritam generiranja ciljnog programa na osnovi troadresnih naredbi. \cite[str.~276-279]{udzbenik}

% 2004 - 3MI - 6 - (7)
\item
Navedite elemente strukture generatora ciljnog programa. \cite[str.~265]{udzbenik}

% 2004 - 3MI - 8 - (7)
\item
Objasnite kako se dobiva i boji graf zavisnosti simboličkih i stvarnih registara te kako se dodjeljuju stvarni registri. \cite[str.~269-273]{udzbenik}

% 2002 - 3MI - 5 - (7)
\item
Za zadani isječak programa odredite izlaz generatora ciljnog programa za Motorolinu seriju procesora 68000.
Na raspolaganju su 3 registra (\texttt{D0}, \texttt{D1} i \texttt{D2}), varijablu \texttt{k} nije moguće pohraniti u registre \texttt{D0} i \texttt{D1}, a memorija se ne smije koristiti.
Pretpostavite da se varijabla \texttt{n} koristi u nastavku programa. 
Ne zahtijeva se uporaba algoritma bojanja grafova.

\begin{alltt}
i=0;
j=3;
k=i+j;
m=k*2;
i=m-5;
j=4*j;
n=j-i;
k=m+i;
\end{alltt}

% 2003 - 3MI - 9 - (7)
\item
Za dani isječak izvornog programa generirajte postfiksni sustav oznaka i pritom definirati korištene operatore grananja.
Na temelju dobivenog postfiksnog sustava oznaka, prikažite sadržaj stoga tijekom generiranja ciljnog programa za neki od Motorolinih mikroprocesora. \cite[str.~256-257, 279-280]{udzbenik}

\begin{alltt}
dok (i>j)
\verb|{|
  ako (i>k)
    k=i+j-m;
  inače
    i=j-4;
\verb|}|
\end{alltt} 

% 2004 - 3MI - 9 - (7)
\item
Za zadanu gramatiku napišite tablice generatora ciljnog programa koji kao ulaz koristi sintaksno stablo.
Ciljni program je strojni jezik ili za Motoroline ili za Intelove procesore.
Operator \texttt{+} označava zbrajanje, a unarni operator \texttt{!} označava logičko invertiranje vrijednosti prema sljedećem pravilu zapisanom u C notaciji: \texttt{!x = (x!=0) ? 0:1}. \cite[str.~281-283]{udzbenik}

\begin{alltt}
<Naredba> \(\to\) <Varijabla> = <Izraz>
<Izraz> \(\to\) <Izraz> + <Izraz>
<Izraz> \(\to\) ! ( <Izraz> )
<Izraz> \(\to\) <Varijabla>
<Varijabla> \(\to\) a | b | c | d | e
\end{alltt} 

% 2009 - 3MI - 5 - (7)
\item
Prikažite tablicu upravljanja za operaciju množenja za generiranje ciljnog programa na osnovi sažetog sintaksnog stabla ako generator raspolaže samo jednim registrom \texttt{R}. \cite[str.~281-283]{udzbenik}

% 2009 - P3MI - 8 - (7)
\item
Na temelju naredbe izvornog programa \texttt{(a+b*c)/(f*g-(d+e)/(h+k))} nacrtajte sažeto sintaksno stablo i na temelju njega tablično prikažite generiranje ciljnog programa. \cite[str.~261-262, 281-284]{udzbenik}

% 2009 - P2MI-L - 5 - (7)
\item
Na temelju naredbe izvornog programa \texttt{a+b*c} nacrtajte sažeto sintaksno stablo i na temelju njega tablično prikažite generiranje ciljnog programa. \cite[str.~261-262, 281-284]{udzbenik}

% 2010 - 3MI - 10 - (7)
\item
Na temelju naredbe izvornog programa \texttt{(a+b*c)/(f-g-h)} nacrtajte sažeto sintaksno stablo i na temelju njega tablično prikažite generiranje ciljnog programa. \cite[str.~261-262, 281-284]{udzbenik}

% 2009 - P3MI - 9 - (7)
\item
Objasnite algoritam generiranja ciljnog programa na temelju troadresnih naredbi.
Prikažite postupak generiranja troadresnih naredbi za sljedeće naredbe izvornog programa: \cite[str.~262-263, 276-279]{udzbenik} \cite{auditorne}

\begin{alltt}
p = (a + b/c) + (d/e)-(d+e) \(\times\) e
r = (p \(\times\) c) + (d \(\times\) e \(\times\) 4)-(d+e) \(\times\) e
\end{alltt} 
%aud/prim 20

% 2009 - P3MI - 10 - (7)
\item
Za zadani programski odsječak primijenite algoritam bojanja grafova kako biste ostvarili pridruživanje registara \texttt{D0}-\texttt{D5}. \cite[str.~269-273]{udzbenik}

\begin{alltt}
ako (j > 3) \verb|{|
   j : = 23;
   i := 11;
   m := j + 5;
\verb|}| inače \verb|{|
   i := 17;
   j := m + 5;
   ako (k < 4) \verb|{|
      k := i + 8;
      m := j – 3;
   \verb|}| inače ako (k == 5) \verb|{|
      k := i;
      i := 19;
      m := i + k * i;
   \verb|}| inače \verb|{|
      m := 1;
   \verb|}|
\verb|}|
i := 4 + 3+m;
\end{alltt} 

\end{enumerate}

%%%%%%%%%%%%%%%%%%%%%%%%%%%%%%%%%%%%%%%%%%%%%%%%%%%%%%%%%%%%%%%%%%%%%%%%%%%%%%%

\chapter{Priprema izvođenja ciljnog programa}

%%%%%%%%%%%%%%%%%%%%%%%%%%%%%%%%%%%%%%%%%%%%%%%%%%%%%%%%%%%%%%%%%%%%%%%%%%%%%%%

\begin{enumerate}[resume]

% 2010 - P3MI - 2 - (8)
% 2010 - P3MI-L - 4 - (8)
\item
Navedite razradbu jezičnih procesora s obzirom na stupanj pripremljenosti ciljnog programa za izvođenje. \cite[str.~286]{udzbenik}

\end{enumerate}

%%%%%%%%%%%%%%%%%%%%%%%%%%%%%%%%%%%%%%%%%%%%%%%%%%%%%%%%%%%%%%%%%%%%%%%%%%%%%%%

\chapter{Optimiranje}

%%%%%%%%%%%%%%%%%%%%%%%%%%%%%%%%%%%%%%%%%%%%%%%%%%%%%%%%%%%%%%%%%%%%%%%%%%%%%%%

\begin{enumerate}[resume]

% 1999 - 3MI - 6 - (9)
\item 
Opišite što se nastoji utvrditi analizom toka podataka. \cite[str.~301-302]{udzbenik}

% 2002 - 2MI - 1 - (9)
\item
Objasnite razliku između strojno nezavisnog i strojno zavisnog optimiranja. \cite[str.~294]{udzbenik}

% 2002 - 3MI - 6 - (9)
\item
Nabrojite komponente koje čine analizu izvođenja programa. \cite[str.~297]{udzbenik}

% 2003 - 3MI - 6 - (9)
\item
Opišite postupak optimiranja petlji kod međukoda niže razine i ciljnog programa. \cite[str.~311-312]{udzbenik}

% 2007 - 3MI - 5 - (9)
\item
Opišite analizu tijeka izvođenja programa. \cite[str.~298-301]{udzbenik}

% 2007 - 3MI - 8 - (9)
\item
Nabrojite i kratko opišite postupke optimiranja međukoda srednje razine. \cite[str.~316-317]{udzbenik}
\end{enumerate}

\newpage
\addcontentsline{toc}{chapter}{Literatura}
\begin{thebibliography}{10}
\bibitem{udzbenik}
Siniša Srbljić:
\emph{Prevođenje programskih jezika}, Element, 2007.
\bibitem{auditorne}
Daniel Skrobo, Ivan Žužak, Miroslav Popović:
\emph{Prevođenje programskih jezika - auditorne vježbe}, ZEMRIS, 2007.
\end{thebibliography}

\end{document}
